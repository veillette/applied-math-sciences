\appendix


\newcommand{\dd}{d}
\renewcommand{\imath}{i}
\newcommand{\sech}{sech}
\newcommand{\csch}{sech}
\newcommand{\arcsinh}{arcsinh}
\newcommand{\arccosh}{arcosh}
\newcommand{\arctanh}{arctanh}
\renewcommand{\imath}{i}
%MAUCH

%% CONTINUE: Monochromatic cow theorem.
%% CONTINUE: 1 = 2.
%% CONTINUE: Buchner Coefficients.
%% CONTINUE: Mauch's theorem.
%% CONTINUE: Unit conversions to beers.

%% CONTINUE: Infinite space Green functions
%% CONTINUE: Green functions for partial differential equations.
%% CONTINUE: update the chapter in definite integrals

%%
%% Add the picture in the complex variables section.
%%
%% Beef up the linear algebra section with some material on matrices.
%% Make sure to have all the background necessary for the Wronskian and tolo.
%%
%% Add the integral of $x \sin x$ etc.
%%
%% Add Bessel functions to the table of Taylor series.
%%
%% Add hyperbolic identities to the trig section.
%%
%% Are the range restrictions in the table of derivatives necessary?
%%
%% Verify the table of derivatives
%%
%% Add a table of differential equations.
%%




%%===========================================================================
%%===========================================================================
\chapter{Greek Letters}
\raggedbottom 

\paragraph{}
The following table shows the greek letters, (some of them have two 
typeset variants), and their corresponding Roman letters.

\setlongtables
\begin{longtable}{l|lll}
  \emph{Name} & \emph{Roman} & \emph{Lower} & \emph{Upper} \\
  \hline
  alpha & a & $\alpha$ & \\
  beta & b & $\beta$ & \\
  chi & c & $\chi$ & \\
  delta & d & $\delta$ & $\Delta$ \\
  epsilon & e & $\epsilon$ & \\
  epsilon (variant) & e & $\varepsilon$ & \\
  phi & f & $\phi$ & $\Phi$ \\
  phi (variant) & f & $\varphi$ & \\
  gamma & g & $\gamma$ & $\Gamma$ \\
  eta & h & $\eta$ & \\
  iota & i & $\iota$ & \\
  kappa & k & $\kappa$ & \\
  lambda & l & $\lambda$ & $\Lambda$ \\
  mu & m & $\mu$ & \\
  nu & n & $\nu$ & \\
  omicron & o & $o$ & \\
  pi & p & $\pi$ & $\Pi$ \\
  pi (variant) & p & $\varpi$ & \\
  theta & q & $\theta$ & $\Theta$ \\
  theta (variant) & q & $\vartheta$ & \\
  rho & r & $\rho$ & \\
  rho (variant) & r & $\varrho$ & \\
  sigma & s & $\sigma$ & $\Sigma$ \\
  sigma (variant) & s & $\varsigma$ & \\
  tau & t & $\tau$ & \\
  upsilon & u & $\upsilon$ & $\Upsilon$ \\
  omega & w & $\omega$ & $\Omega$ \\
  xi & x & $\xi$ & $\Xi$ \\
  psi & y & $\psi$ & $\Psi$ \\
  zeta & z & $\zeta$ & 
\end{longtable}















\raggedbottom
%%===========================================================================
%%===========================================================================
\chapter{Table of Derivatives}
\label{table_of_derivatives}
\raggedbottom 

%% CONTINUE add page references

Note: $c$ denotes a constant and $'$ denotes differentiation.

\setlongtables
\begin{longtable}{l}
  $\displaystyle \frac{\dd}{\dd x} (f g) = \frac{\dd f}{\dd x} g + f \frac{\dd g}{\dd x}$\\
  \\
  $\displaystyle \frac{\dd}{\dd x} \, \frac{f}{g}
  = \frac{f' g - f g'}{g^2}$ \\
  \\
  $\displaystyle \frac{\dd}{\dd x} f^c = c f^{c-1}f'$ \\
  \\
  $\displaystyle \frac{\dd}{\dd x} f(g) = f'(g) g'$ \\
  \\
  $\displaystyle \frac{\dd^2}{\dd x^2} f(g) = f''(g) (g')^2 + f' g''$ \\
  \\
  $\displaystyle \frac{\dd^n}{\dd x^n}(f g) 
  = \binom{n}{0} \frac{\dd^n f}{\dd x^n} g + 
  \binom{n}{1} \frac{\dd^{n-1} f}{\dd x^{n-1}} \frac{\dd g}{\dd x} + 
  \binom{n}{2} \frac{\dd^{n-2} f}{\dd x^{n-2}} \frac{\dd^2 g}{\dd x^2} + \cdots + 
  \binom{n}{n} f \frac{\dd^n g}{\dd x^n}$ \\
  \\
  $\displaystyle \frac{\dd}{\dd x} \ln x = \frac{1}{|x|}$ \\
  \\
  $\displaystyle \frac{\dd}{\dd x} c^x = c^x \ln c$ \\
  \\
  $\displaystyle \frac{\dd}{\dd x} f^g = g f^{g-1} \frac{\dd f}{\dd x}
  + f^g \ln f \frac{\dd g}{\dd x}$ \\
  \\
  $\displaystyle \frac{\dd}{\dd x} \sin x = \cos x$ \\
  \\
  $\displaystyle \frac{\dd}{\dd x} \cos x = - \sin x$ \\
  \\
  $\displaystyle \frac{\dd}{\dd x} \tan x = \sec^2 x$ \\
  \\
  $\displaystyle \frac{\dd}{\dd x} \csc x = - \csc x \cot x$ \\
  \\
  $\displaystyle \frac{\dd}{\dd x} \sec x = \sec x \tan x$ \\
  \\
  $\displaystyle \frac{\dd}{\dd x} \cot x = - \csc^2 x$ \\ 
  \\
  $\displaystyle \frac{\dd}{\dd x} \arcsin x = \frac{1}{\sqrt{1-x^2}}, \qquad
  -\frac{\pi}{2} \leq \arcsin x \leq \frac{\pi}{2}$ \\
  \\
  $\displaystyle \frac{\dd}{\dd x} \arccos x = - \frac{1}{\sqrt{1-x^2}}, \qquad
  0 \leq \arccos x \leq \pi$ \\
  \\
  $\displaystyle \frac{\dd}{\dd x} \arctan x = \frac{1}{1+x^2}, \qquad
  -\frac{\pi}{2} \leq \arctan x \leq \frac{\pi}{2}$ \\
  \\
  $\displaystyle \frac{\dd}{\dd x} \sinh x = \cosh x$ \\
  \\
  $\displaystyle \frac{\dd}{\dd x} \cosh x = \sinh x$ \\
  \\
  $\displaystyle \frac{\dd}{\dd x} \tanh x = \sech^2 x$ \\
  \\
  $\displaystyle \frac{\dd}{\dd x} \csch x = - \csch x \coth x$ \\
  \\
  $\displaystyle \frac{\dd}{\dd x} \sech x = - \sech x \tanh x$ \\
  \\
  $\displaystyle \frac{\dd}{\dd x} \coth x = - \csch^2 x$ \\
  \\
  $\displaystyle \frac{\dd}{\dd x} \arcsinh x = \frac{1}{\sqrt{x^2+1}}$ \\
  \\
  $\displaystyle \frac{\dd}{\dd x} \arccosh x = \frac{1}{\sqrt{x^2-1}}, \qquad
  x > 1,\ \arccosh x > 0$ \\
  \\
  $\displaystyle \frac{\dd}{\dd x} \arctanh x = \frac{1}{1-x^2}, \qquad
  x^2 < 1$ \\
  \\
  $\displaystyle \frac{\dd}{\dd x} \int_c^x f(\xi)\,d\xi = f(x)$ \\
  \\
  $\displaystyle \frac{\dd}{\dd x} \int_x^c f(\xi)\,d\xi = - f(x)$ \\
  \\
  $\displaystyle \frac{\dd}{\dd x} \int_g^h f(\xi,x)\,d\xi 
  = \int_g^h \frac{\partial f(\xi,x)}{\partial x}\,d\xi + f(h,x) h' - f(g,x) g'$
\end{longtable}










\raggedbottom
%%===========================================================================
%%===========================================================================
\chapter{Table of Integrals}
\label{table_of_integrals}
\raggedbottom 

\setlongtables
\begin{longtable}{l}
  $\displaystyle \int u \frac{\dd v}{\dd x}\,d x = u v - \int v \frac{\dd u}{\dd x}\,d x$ \\
  \\
  $\displaystyle \int \frac{f'(x)}{f(x)}\,d x = \log f(x)$ \\
  \\
  $\displaystyle \int \frac{f'(x)}{2 \sqrt{f(x)}}\,d x = \sqrt{f(x)}$ \\
  \\
  $\displaystyle \int x^\alpha\,d x = \frac{x^{\alpha+1}}{\alpha+1} 
  \qquad \text{for }\alpha \neq = -1$ \\
  \\
  $\displaystyle \int \frac{1}{x}\,d x = \log x$ \\
  \\
  $\displaystyle \int e^{a x}\,d x = \frac{e^{a x}}{a}$ \\
  \\
  $\displaystyle \int a^{b x}\,d x = \frac{a^{b x}}{b \log a} 
  \qquad \text{for }a > 0$ \\
  \\
  $\displaystyle \int \log x\,d x = x \log x - x$ \\
  \\
  $\displaystyle \int \frac{1}{x^2 + a^2}\,d x 
  = \frac{1}{a} \arctan \frac{x}{a}$ \\
  \\
  $\displaystyle \int \frac{1}{x^2-a^2}\,d x = 
  \begin{cases}
    \frac{1}{2a} \log \frac{a-x}{a+x} \quad &\text{for }x^2 < a^2 \\
    \frac{1}{2a} \log \frac{x-a}{x+a} \quad &\text{for }x^2 > a^2
  \end{cases}$ \\
  \\
  $\displaystyle \int \frac{1}{\sqrt{a^2-x^2}}\,d x = \arcsin \frac{x}{|a|} 
  = - \arccos \frac{x}{|a|} \qquad \text{for } x^2 < a^2$ \\
  \\
  $\displaystyle \int \frac{1}{\sqrt{x^2 \pm a^2}}\,d x 
  = \log (x + \sqrt{x^2 \pm a^2})$ \\
  \\
  $\displaystyle \int \frac{1}{x \sqrt{x^2 - a^2}}\,d x 
  = \frac{1}{|a|} \sec^{-1} \frac{x}{a}$ \\
  \\
  $\displaystyle \int \frac{1}{x \sqrt{a^2 \pm x^2}}\,d x 
  = - \frac{1}{a} \log \left(
    \frac{a + \sqrt{a^2 \pm x^2}}{x} \right)$ \\
  \\
  $\displaystyle \int \sin(a x)\,d x = - \frac{1}{a} \cos (a x)$ \\
  \\
  $\displaystyle \int \cos(a x)\,d x = \frac{1}{a} \sin (a x)$ \\
  \\
  $\displaystyle \int \tan(a x)\,d x = - \frac{1}{a} \log \cos (a x)$ \\
  \\
  $\displaystyle \int \csc(a x)\,d x = \frac{1}{a} \log \tan \frac{a x}{2}$ \\
  \\
  $\displaystyle \int \sec(a x)\,d x 
  = \frac{1}{a} \log \tan \left( \frac{\pi}{4} + \frac{a x}{2}
  \right)$\\
  \\
  $\displaystyle \int \cot(a x)\,d x = \frac{1}{a} \log \sin (a x)$ \\
  \\
  $\displaystyle \int \sinh(a x)\,d x = \frac{1}{a} \cosh (a x)$ \\
  \\
  $\displaystyle \int \cosh(a x)\,d x = \frac{1}{a} \sinh (a x)$ \\
  \\
  $\displaystyle \int \tanh(a x)\,d x = \frac{1}{a} \log \cosh (a x)$ \\
  \\
  $\displaystyle \int \csch(a x)\,d x = \frac{1}{a} \log \tanh \frac{a x}{2}$ \\
  \\
  $\displaystyle \int \sech(a x)\,d x 
  = \frac{i}{a} \log \tanh \left( \frac{i \pi}{4} +
    \frac{a x}{2} \right)$ \\
  \\
  $\displaystyle \int \coth(a x)\,d x = \frac{1}{a} \log \sinh (a x)$ \\
  \\
  $\displaystyle \int x \sin a x\,d x 
  = \frac{1}{a^2} \sin a x - \frac{x}{a} \cos a x$ \\
  \\
  $\displaystyle \int x^2 \sin a x\,d x 
  = \frac{2x}{a^2} \sin a x - \frac{a^2 x^2-2}{a^3} \cos a x$ \\
  \\
  $\displaystyle \int x \cos a x\,d x 
  = \frac{1}{a^2} \cos a x + \frac{x}{a} \sin a x$ \\
  \\
  $\displaystyle \int x^2 \cos a x\,d x 
  = \frac{2x \cos a x}{a^2} + \frac{a^2 x^2 - 2}{a^3} \sin a x$
\end{longtable}






\raggedbottom



%%===============================
\chapter{Algebra}
\label{algebra_appendix}
\index{algebra_appendix}
\raggedbottom 
\section{Quadratic equation}

  The solutions of $ax^2+bx+c=0$ \\
  are 
\begin{equation*}
x=\frac{-b\pm\sqrt{b^2-4ac}}{2a}
\end{equation*}


\section{Logarithms and exponentials}

  \begin{equation*}   \ln(ab)=\ln a + \ln b    \end{equation*}
  \begin{equation*}   e^{a+b} = e^ae^b    \end{equation*}
  \begin{equation*}   \ln e^x = e^{\ln x} = x    \end{equation*}
  \begin{equation*}   \ln(a^b) = b \ln a    \end{equation*}


\section{Geometry, area, and volume}

\noindent\begin{tabular}{ll}
  area of a triangle of base $b$ and height $h$     & = $\frac{1}{2}bh$ \\
  circumference of a circle of radius $r$           &= $2\pi r$ \\
  area of a circle of radius $r$                    &= $\pi r^2$ \\
  surface area of a sphere of radius $r$            &= $4\pi r^2$ \\
  volume of a sphere of radius $r$                  &= $\frac{4}{3}\pi r^3$
\end{tabular}

\section{Trigonometry with a right triangle}

\newcommand{\tikzAngleOfLine}{\tikz@AngleOfLine}
  \def\tikz@AngleOfLine(#1)(#2)#3{%
  \pgfmathanglebetweenpoints{%
    \pgfpointanchor{#1}{center}}{%
    \pgfpointanchor{#2}{center}}
  \pgfmathsetmacro{#3}{\pgfmathresult}%
  }
\begin{figure}
\begin{center}  \begin{tikzpicture}
    \coordinate (A) at (4,3);
    \coordinate (C) at (4,0);
    \coordinate (B) at (0,0);
\draw (A) node[left]{} -- (B) node[midway,above]{$C$} -- (C)node[midway,below]{$A$} -- (A)node[midway,right]{$B$};
    \tikzAngleOfLine(B)(A){\AngleStartAB}
    \tikzAngleOfLine(B)(C){\AngleEndAC}

  \draw[red,-] (B)+(\AngleStartAB:1.5cm) arc (\AngleStartAB:\AngleEndAC:1.5 cm);
    \node[circle] at ($(B)+({(\AngleStartAB+\AngleEndAC)/2}:1 cm)$) {$\theta$};

\end{tikzpicture}
\end{center}
\caption{ Triangle}
\end{figure}


  \begin{equation*}
 \sin\theta  =B /C \quad
 \cos\theta = A/C \quad
 \tan\theta = B/A
 \end{equation*}

Pythagorean theorem: $C^2=A^2+B^2$

\section{Trigonometry with any triangle}


%\newcommand{\tikzAngleOfLine}{\tikz@AngleOfLine}
%  \def\tikz@AngleOfLine(#1)(#2)#3{%
%  \pgfmathanglebetweenpoints{%
%    \pgfpointanchor{#1}{center}}{%
%    \pgfpointanchor{#2}{center}}
%  \pgfmathsetmacro{#3}{\pgfmathresult}%
%  }
\begin{figure}
\begin{center}  \begin{tikzpicture}
    \coordinate (A) at (1,1);
    \coordinate (B) at ($(A)+(25:3)$);
    \coordinate (C) at ($(A)+(100:5)$);

\draw (A) node[left]{$a$} -- (B) node[right]{$b$}node[midway,below]{$C$} -- (C)node[above]{$c$}node[midway,above]{$A$} -- (A)node[midway,left]{$B$};

    \tikzAngleOfLine(A)(B){\AngleStartAB}
    \tikzAngleOfLine(A)(C){\AngleEndAC}

    \tikzAngleOfLine(B)(A){\AngleStartBA}
    \tikzAngleOfLine(B)(C){\AngleEndBC}

    \tikzAngleOfLine(C)(A){\AngleStartCA}
    \tikzAngleOfLine(C)(B){\AngleEndCB}


    \draw[red,-] (A)+(\AngleStartAB:1cm) arc (\AngleStartAB:\AngleEndAC:1 cm);
    \node[circle] at ($(A)+({(\AngleStartAB+\AngleEndAC)/2}:0.5 cm)$) {$\alpha$};

    \draw[red,-] (B)+(\AngleStartBA:1cm) arc (\AngleStartBA:\AngleEndBC:1 cm);
    \node[circle] at ($(B)+({(\AngleStartBA+\AngleEndBC)/2}:0.5 cm)$) {$\beta$};

    \draw[red,-] (C)+(\AngleStartCA:1cm) arc (\AngleStartCA:\AngleEndCB:1 cm);
    \node[circle] at ($(C)+({(\AngleStartCA+\AngleEndCB)/2}:0.5 cm)$) {$\gamma$};

\end{tikzpicture}
\end{center}
\caption{ Triangle}
\end{figure}

Law of Sines:
  \begin{equation*} \frac{\sin\alpha}{A}=\frac{\sin\beta}{B}=\frac{\sin\gamma}{C} \end{equation*}

\noindent Law of Cosines:
  \begin{equation*} C^2 = A^2 + B^2 - 2AB \cos \gamma \end{equation*}


%%============================================================================
\chapter{Trigonometric Identities}
\label{trigonometric_identities}
\index{trigonometric identities}
\raggedbottom 
\section{Circular Functions}

\paragraph{Pythagorean Identities}
\[      \sin^2 x + \cos^2 x = 1, \qquad
1 + \tan^2 x = \sec^2 x, \qquad
1 + \cot^2 x = \csc^2 x \]

\paragraph{Angle Sum and Difference Identities}
\begin{align*}
  &\sin(x+y) = \sin x \cos y + \cos x \sin y \\
  &\sin(x-y) = \sin x \cos y - \cos x \sin y \\
  &\cos(x+y) = \cos x \cos y - \sin x \sin y \\
  &\cos(x-y) = \cos x \cos y + \sin x \sin y 
\end{align*}

\paragraph{Function Sum and Difference Identities}
\begin{align*}
  &\sin x + \sin y = 2 \sin \frac{1}{2}(x+y) \cos \frac{1}{2}(x-y) \\
  &\sin x - \sin y = 2 \cos \frac{1}{2}(x+y) \sin \frac{1}{2}(x-y) \\
  &\cos x + \cos y = 2 \cos \frac{1}{2}(x+y) \cos \frac{1}{2}(x-y) \\
  &\cos x - \cos y =-2 \sin \frac{1}{2}(x+y) \sin \frac{1}{2}(x-y) 
\end{align*}


\paragraph{Double Angle Identities}
\[\sin 2x = 2 \sin x \cos x, \qquad \cos 2x = \cos^2 x - \sin^2 x \]


\paragraph{Half Angle Identities}
\[\sin^2 \frac{x}{2} = \frac{1-\cos x}{2}, \qquad
\cos^2 \frac{x}{2} = \frac{1+\cos x}{2} \]

\paragraph{Function Product Identities}
\begin{align*}
  &\sin x \sin y = \frac{1}{2} \cos(x-y) - \frac{1}{2} \cos(x+y) \\
  &\cos x \cos y = \frac{1}{2} \cos(x-y) + \frac{1}{2} \cos(x+y) \\
  &\sin x \cos y = \frac{1}{2} \sin(x+y) + \frac{1}{2} \sin(x-y) \\
  &\cos x \sin y = \frac{1}{2} \sin(x+y) - \frac{1}{2} \sin(x-y) 
\end{align*}

\paragraph{Exponential Identities}
\[      
e^{\imath x} = \cos x + \imath \sin x, \qquad
\sin x = \frac{e^{\imath x} - e^{-\imath x}}{\imath 2}, \qquad
\cos x = \frac{e^{\imath x} + e^{-\imath x}}{2} 
\]

%%---------------------------------------------------------------------------
\section{Hyperbolic Functions}

\paragraph{Exponential Identities}
\[ \sinh x = \frac{e^x - e^{-x}}{2}, \qquad \cosh x = \frac{e^x + e^{-x}}{2}\]
\[ \tanh x = \frac{\sinh x}{\cosh x} = \frac{e^x - e^{-x}}{e^x + e^{-x}} \]

\paragraph{Reciprocal Identities}
\[ \csch x = \frac{1}{\sinh x}, \qquad \sech x = \frac{1}{\cosh x}, \qquad
\coth x = \frac{1}{\tanh x} \]

\paragraph{Pythagorean Identities}
\[ \cosh^2 x - \sinh^2 x = 1, \qquad \tanh^2 x + \sech^2 x = 1 \]

\paragraph{Relation to Circular Functions}
\begin{alignat*}{2}
  \sinh(\imath x) &= \imath \sin x &\qquad \sinh x &= -\imath \sin(\imath x)
  \\
  \cosh(\imath x) &= \cos x &\qquad \cosh x &= \cos(\imath x)
  \\
  \tanh(\imath x) &= \imath \tan x &\qquad \tanh x &= - \imath \tan(\imath x)
\end{alignat*}

\paragraph{Angle Sum and Difference Identities}
\begin{align*}
  &\sinh(x \pm y) = \sinh x \cosh y \pm \cosh x \sinh y \\
  &\cosh(x \pm y) = \cosh x \cosh y \pm \sinh x \sinh y \\
  &\tanh(x \pm y) = \frac{\tanh x \pm \tanh y}{1 \pm \tanh x \tanh y}
  = \frac{\sinh 2x \pm \sinh 2y}{\cosh 2x \pm \cosh 2y} \\
  &\coth(x \pm y) = \frac{1 \pm \coth x \coth y}{\coth x \pm \coth y}
  = \frac{\sinh 2x \mp \sinh 2y}{\cosh 2x - \cosh 2y}
\end{align*}

\paragraph{Function Sum and Difference Identities}
\begin{align*}
  &\sinh x \pm \sinh y = 2 \sinh \frac{1}{2}(x \pm y) 
  \cosh \frac{1}{2}(x \mp y) \\
  &\cosh x + \cosh y = 2 \cosh \frac{1}{2}(x+y) \cosh \frac{1}{2}(x-y) \\
  &\cosh x - \cosh y = 2 \sinh \frac{1}{2}(x+y) \sinh \frac{1}{2}(x-y) \\
  &\tanh x \pm \tanh y = \frac{\sinh(x \pm y)}{\cosh x \cosh y} \\
  &\coth x \pm \coth y = \frac{\sinh(x \pm y)}{\sinh x \sinh y} \\
\end{align*}


\paragraph{Double Angle Identities}
\[\sinh 2x = 2 \sinh x \cosh x, \qquad \cosh 2x = \cosh^2 x + \sinh^2 x \]


\paragraph{Half Angle Identities}
\[\sinh^2 \frac{x}{2} = \frac{\cosh x - 1}{2}, \qquad
\cosh^2 \frac{x}{2} = \frac{\cosh x + 1}{2} \]

\paragraph{Function Product Identities}
\begin{align*}
  &\sinh x \sinh y = \frac{1}{2} \cosh(x+y) - \frac{1}{2} \cosh(x-y) \\
  &\cosh x \cosh y = \frac{1}{2} \cosh(x+y) + \frac{1}{2} \cosh(x-y) \\
  &\sinh x \cosh y = \frac{1}{2} \sinh(x+y) + \frac{1}{2} \sinh(x-y) \\
\end{align*}

See Figure~\ref{sicotanh} for plots of the hyperbolic
circular functions.

\begin{figure}[h!]
  \begin{center}
    \includegraphics[height=1.5in]{appendix/sicotanh}
%\begin{tikzpicture}
%\begin{axis}[xmin=-3,xmax=3,grid=major,legend entries={$\cosh$,[red]$\sinh$,$\tanh$}]]
%\addplot [color=red,mark=none,smooth] {(exp(x)+exp(-x))/2};
%\addplot [color=blue,mark=none,smooth] {(exp(x)-exp(-x))/2};
%\addplot [color=green,mark=none,smooth]{(exp(x)-exp(-x))/(exp(x)+exp(-x))};
%\end{axis}
%\end{tikzpicture}
  \end{center}
  \caption{$\cosh x$, $\sinh x$ and then $\tanh x$}
  \label{sicotanh}
\end{figure}





\raggedbottom
%%===========================================================================
\chapter{Formulas from Linear Algebra}
\label{appendix formulas from linear algebra}
\flushbottom

\paragraph{Kramer's Rule.}
\index{Kramer's rule}

Consider the matrix equation
\[ A \vec{x} = \vec{b}. \]
This equation has a unique solution if and only if $\det(A) \neq 0$.  
If the determinant vanishes then there are either no solutions or an 
infinite number of solutions.
If the determinant is nonzero, the solution for each $x_j$ can be written
\[ x_j = \frac{\det A_j}{\det A} \]
where $A_j$ is the matrix formed by replacing the $j^{t h}$ column of $A$ 
with $b$.


  The matrix equation
  \[ \begin{pmatrix}
    1       &       2 \\
    3       &       4 
  \end{pmatrix}
  \begin{pmatrix}
    x_1 \\
    x_2
  \end{pmatrix}
  = 
  \begin{pmatrix}
    5 \\
    6
  \end{pmatrix},
  \]
  has the solution
  \[ x_1 = \frac{
    \begin{vmatrix}
      5       &       2 \\
      6       &       4
    \end{vmatrix} }{
    \begin{vmatrix}
      1       &       2 \\
      3       &       4
    \end{vmatrix} }
  = \frac{8}{-2} = -4, \qquad
  x_2 = \frac{
    \begin{vmatrix}
      1       &       5 \\
      3       &       6
    \end{vmatrix} }{
    \begin{vmatrix}
      1       &       2 \\
      3       &       4
    \end{vmatrix} }
  = \frac{-9}{-2} = \frac{9}{2}.
  \]













\raggedbottom
%%=============================================================================
\chapter{Vector Analysis}
\flushbottom

\paragraph{Rectangular Coordinates}

\[
f = f(x,y,z), \qquad \vec{g} = g_x \mathbf{i} + g_y \mathbf{j} + g_z \mathbf{k}
\]

\[
\nabla f = \frac{\partial f}{\partial x} \mathbf{i} + \frac{\partial f}{\partial y} \mathbf{j} + \frac{\partial f}{\partial z} \mathbf{k}
\]

\[
\nabla \cdot \vec{g} = \frac{\partial g_x}{\partial x} + \frac{\partial g_y}{\partial y}
+ \frac{\partial g_z}{\partial z}
\]

\[
\nabla \times \vec{g} =
\begin{vmatrix}
  \mathbf{i} & \mathbf{j} & \mathbf{k} \\
  \frac{\partial}{\partial x} & \frac{\partial}{\partial y} & \frac{\partial}{\partial z} \\
  g_x & g_y & g_z
\end{vmatrix}
\]

\[
\Delta f = \nabla^2 f = \frac{\partial^2 f}{\partial x^2} + \frac{\partial^2 f}{\partial y^2} + \frac{\partial^2 f}{\partial z^2}
\]


\paragraph{Spherical Coordinates}

\[
x = r \cos \theta \sin \phi, \qquad
y = r \sin \theta \sin \phi, \qquad
z = r \cos \phi
\]

\[
f = f(r, \theta, \phi), \qquad \vec{g} = 
g_r \mathbf{r} + g_\theta \boldsymbol{\theta} + g_\phi \boldsymbol{\phi}
\]


%%CONTINUE

%%\paragraph{Cylindrical Coordinates}


%%CONTINUE



\paragraph{Divergence Theorem.}
\[
\iint \nabla \cdot \mathbf{u} \,\dd x \,\dd y 
= \oint \mathbf{u} \cdot \mathbf{n} \,\dd s
\]

\paragraph{Stoke's Theorem.}
\[
\iint (\nabla \times \mathbf{u}) \cdot \,\dd \mathbf{s}
= \oint \mathbf{u} \cdot \,\dd \mathbf{r}
\]

%% CONTINUE






\raggedbottom
%%=============================================================================
\chapter{Physics}
\label{chapter spherical chicken}
\index{spherical chicken}
\index{chicken!spherical}
\flushbottom


In order to reduce processing costs, a chicken farmer wished to acquire
a plucking machine.  Since there was no such machine on the market, 
he hired a mechanical engineer to design one.  After extensive
research and testing, the professor concluded that it was impossible to 
build such a machine with current technology.  The farmer was disappointed,
but not wanting to abandon his dream of an automatic plucker, he consulted
a physicist.  After a single afternoon of work, the physicist reported 
that not only could a plucking machine be built, but that the design 
was simple.  The elated farmer asked him to describe his method.
The physicist replied, ``First, assume a spherical chicken \ldots''.

The problems in this text will implicitly make certain simplifying 
assumptions about chickens.  For example, a problem might assume 
a perfectly elastic, frictionless, spherical chicken.  In two-dimensional
problems, we will assume that chickens are circular.



\chapter{Sophisms}
\label{chapter  proofs}
\index{sophisms}
\flushbottom


Let $x$ and $y$ be equal non-zero quantities
\begin{equation*}
x=y
\end{equation*}
Multiply through by $x$
\begin{equation*}
x^2 = x y 
\end{equation*}
 Subtract 
 \begin{equation*}
x^2 - y^2 = x y - y^2 
\end{equation*}
Factor both sides
\begin{equation*}
(x - y)(x + y) = y (x - y) 
\end{equation*}
 Divide out
 \begin{equation*}
\frac{(x - y)(x + y)}{(x - y)} = \frac{y(x - y)}{(x - y)} 
\end{equation*}
 Cancel out 
\begin{equation*}
x + y = y 
\end{equation*}
Observing that  $x=y$
\begin{equation*}
y + y = y 
\end{equation*}
Combine like terms on the left
\begin{equation*}
2 y =y 
\end{equation*}
Divide by the non-zero $y$ 
\begin{equation*}
2=1
\end{equation*}
\begin{flushright}
 Q.E.D.
\end{flushright}

Also consider
\begin{equation*}
1 = \sqrt{1} = \sqrt{(-1)(-1)} = \sqrt{-1}\sqrt{-1}=i \cdot i = -1.
\end{equation*}
\raggedbottom
%%=============================================================================
\chapter{Glossary}
\flushbottom


\paragraph{}

The poet-philosopher Steve Martin once said:
\begin{quotation}
``Oeuf'' means egg, ``chapeau'' means hat. It's like those French
have a different word for everything.
\end{quotation}

Phrases often have different meanings in mathematics than in everyday usage.
Here I have collected definitions of some mathematical terms which might
confuse the novice.


\begin{description}
\item{\textbf{beyond the scope of this text:}}
  Beyond the comprehension of the author.
\item{\textbf{difficult:}}
  Essentially impossible.  Note that mathematicians never refer to problems 
  they have solved as being difficult.  This would either be boastful, 
  (claiming that you can solve difficult problems), or self-deprecating,
  (admitting that you found the problem to be difficult).
\item{\textbf{interesting:}}
  This word is grossly overused in math and science.
  It is often used to describe any work that the author has done, regardless 
  of the work's significance or novelty.
  It may also be used as a synonym for difficult.
  It has a completely different meaning when used by the non-mathematician.
  When I tell people that I am a mathematician they typically respond 
  with, ``That must be interesting.'', which means, ``I don't know 
  anything about math or what mathematicians do.''  I typically answer,
  ``No.  Not really.''  
\item{\textbf{non-obvious} or \textbf{non-trivial:}}
  Nearly impossible.
\item{\textbf{one can prove that \ldots:}}
  The ``one'' that proved it was a genius like Gauss.  The phrase literally 
  means ``you haven't got a chance in hell of proving that \ldots''
\item{\textbf{simple:}}
  Mathematicians communicate their prowess to colleagues and students by 
  referring to all problems as simple or trivial.  If you ever become a math
  professor, introduce every example as being ``really quite trivial.''
  \footnote{For even more fun say it in your best Elmer Fudd accent.
    ``This next pwobwem is weawy quite twiviaw''.}
\end{description}



\paragraph{}
Here are some less interesting words and phrases that you are 
probably already familiar with.

\begin{description}
\item{\textbf{corollary:}}
  a proposition inferred immediately from a proved proposition with little 
  or no additional proof
\item{\textbf{lemma:}}
  an auxiliary proposition used in the demonstration of another proposition
\item{\textbf{theorem:}}
  a formula, proposition, or statement in mathematics or logic deduced or 
  to be deduced from other formulas or propositions
\end{description}


\chapter{Notation and Units}
\label{notationtable}

\section{SI base units}

\noindent\begin{tabular}{|l|l|l|}
\hline
quantity	& unit	& symbol \\
\hline
distance	& meter, m	& $d, L, x, \Delta{}x$ \\
time	& second, s	& $t, \Delta{}t$ \\
mass	& kilogram, kg	& $m$ \\
temperature	& kelvin, K	& $T$ \\
amount of substance & mole, mol & $N$ \\
electric current & ampere, A & $I$ \\
luminous intensity & candela, cd & $I$ \\
\hline
\end{tabular}

\section{SI derived units}

\noindent\begin{tabular}{|l|l|l|}
\hline
quantity	& unit	& typical symbol \\
\hline
distance	& meter, m	& $x, \Delta{}x$ \\
time	& second, s	& $t, \Delta{}t$ \\
mass	& kilogram, kg	& $m$ \\
density	& $\kgunit/\munit^3$	& $\rho$  \\
force	& newton, 1 N=$1\ \kgunit\unitdot\munit/\sunit^2$	& $F$ \\
velocity	& m/s	& $v$ \\
acceleration	& $\munit/\sunit^2$	& $a$ \\
gravitational field	& $\junit/\kgunit\unitdot\munit$ or $\munit/\sunit^2$	& $g$ \\
energy	& joule, J	& $E$ (also electric field)\\
momentum	& $\kgunit\unitdot\munit/\sunit$	& $p$ \\
angular momentum	& $\kgunit\unitdot\munit^2/\sunit$ or $\junit\unitdot\sunit$	& $L$ (also inductance)\\
power	& watt, 1 W = 1 J/s	& $P$ (also pressure) \\
pressure & 1 Pa=$1\ \nunit/\munit^2$	& $P$ (also power)\\

period	& s	& $T$ (also temperature)\\
wavelength	& m	& $\lambda$ \\
frequency	& $\zu{s}^{-1}$ or Hz	& $f$ \\
charge	& coulomb, C	& $q$ \\
voltage	& volt, 1 V = 1 J/C	& $V$ \\
current	& ampere, 1 A = 1 C/s	& $I$ \\
resistance	& ohm, 1 $\Omega$ = 1 V/A	& $R$ \\
capacitance	& farad, 1 F = 1 C/V	& $C$ \\
inductance	& henry, 1 H = 1 $\zu{V}\unitdot\sunit/\zu{A}$	& $L$ (also angular momentum)\\
electric field	& V/m or N/C	& $E$ (also energy)\\
magnetic field	& tesla, 1 T = 1 $\nunit\unitdot\sunit/\zu{C}\unitdot\munit$	& $B$ \\
focal length	& m	& $f$ \\
magnification	& unitless	& $M$ \\
index of refraction	& unitless	& $n$ \\
electron wavefunction	& $\munit^{-3/2}$	& $\Psi$ \\
\hline
\end{tabular}
 
\chapter{Fundamental constants}
\noindent\begin{tabular}{|l|l|}
\hline
gravitational constant	& $G=6.67\times10^{-11}\ \junit\unitdot\munit/\kgunit^2$ \\
Boltzmann constant      & $k=1.38\times10^{-23}\ \junit/\kunit$\\
Coulomb constant	& $k=8.99\times10^9\ \junit\unitdot\munit/\zu{C}^2$ or $\nunit\unitdot\munit^2/\zu{C}^2$ \\
quantum of charge	& $e=1.60\times10^{-19}\ \zu{C}$ \\
speed of light	& $c=3.00\times10^8\ \zu{m/s}$ \\
Planck's constant	& $h=6.63\times10^{-34}\ \junit\unitdot\sunit$ \\
\hline
\end{tabular}

\noindent Note the use of the same notation, $k$, for both the Boltzmann constant and the Coulomb constant.

\section{Subatomic particles}\label{subatomicparticlesdata}
\noindent\begin{tabular}{|l|l|l|l|}
\hline
particle	& mass (kg)	& charge	& radius (fm) \\
\hline
electron	& $9.109\times10^{-31}$	& $-e$	& $\lesssim0.01$\\
proton	& $1.673\times10^{-27}$	& $+e$	& $\sim{}1.1$\\
neutron	& $1.675\times10^{-27}$	& 0		& $\sim{}1.1$\\
neutrino	& $\sim10^{-39}$ kg ?	& 0		& 	?\\
\hline
\end{tabular}

\noindent{}The radii of protons and neutrons can only be given
approximately, since they have fuzzy surfaces. For
comparison, a typical atom is about a million fm in radius.
 %




\section{Earth, moon, and sun}
\noindent\begin{tabular}{|l|l|l|l|}
\hline
body		&	mass (kg)		&	radius (km)	&	radius of orbit (km) \\
\hline
earth	&	$5.97\times10^{24}$	&	$6.4\times10^{3}$	&	$1.49\times10^{8}$\\
moon	&	$7.35\times10^{22}$	&	$1.7\times10^{3}$	&	$3.84\times10^{5}$\\
sun		&	$1.99\times10^{30}$	&	$7.0\times10^{5}$	&	---\\
\hline
\end{tabular}
 %

 \chapter{Metric prefixes}\label{metricprefixestable}\index{metric system!prefixes}
 \noindent\begin{tabular}{|l|l|l|}
 \hline

Symbol & Prefix & $10^n$ \\
\hline
P & peta & $10^{15}$ \\
 T & tera & $10^{12}$ \\
 G     & giga                  & $10^9$ \\
 M	& mega			& $10^6$ \\
 k	& kilo			& $10^3$ \\
 m	& milli		& $10^{-3}$ \\
 $\mu$- (Greek mu) & micro	& $10^{-6}$ \\
 n	& nano			& $10^{-9}$ \\
 p	& pico			& $10^{-12}$ \\
 f	& femto		& $10^{-15}$ \\
 \hline
 \end{tabular}

\noindent{}
Each prefix name has an associated symbol which can be used in combination with the symbols for units of measure. Thus, the ``kilo-'' symbol, k, can be used to produce km, kg, and kW, (kilometre, kilogram, and kilowatt).

 \noindent{}Note that the exponents go in steps of three.
 The exception is centi-, $10^{-2}$, which is used only in the centimeter, and this
 doesn't require memorization, because a cent is $10^{-2}$ dollars.
%  %

 \chapter{Nonmetric units}\label{nonmetricunits}\index{units!nonmetric}
 \noindent Nonmetric units in terms of metric ones:\\
 \noindent\begin{tabular}{|l|l|}
 \hline
 1 inch	&= 25.4 mm (by definition)\\
 1 pound (lb)	&= 4.5 newtons of force\\
 1 scientific calorie &= 4.18 J\\
 1 nutritional calorie &= $4.18\times10^3$ J\\
 1 gallon &= $3.78\times10^3\ \zu{cm}^3$\\
 1 horsepower &= 746 W\\
 \hline
 \end{tabular}

 \noindent{}The pound is a unit of force, so it converts to newtons, not kilograms.
 A one-kilogram mass at the earth's surface experiences a gravitational force of
 $(1\ \kgunit)(9.8\ \munit/\sunit^2)=9.8\ \zu{N}=2.2\ \zu{lb}$. The nutritional
 information on food packaging
 typically gives energies in units of calories, but those so-called calories are
 really kilocalories.

 \noindent Relationships among U.S. units:\\
 \noindent\begin{tabular}{|l|l|}
 \hline
 1 foot (ft)	&= 12 inches\\
 1 yard (yd) &= 3 feet \\
 1 mile (mi) &= 5280 feet\\
 1 ounce (oz) &= 1/16 pound\\
 \hline
\end{tabular}

 \raggedbottom

\chapter{Graphing}
%Begin Section 5.1
The trigonometric functions can be graphed just like any other function, as we will now show.
In the graphs we will always use radians for the angle measure.\index{graphs}\index{trigonometric
functions!graphs of}

\section{Graphing the Trigonometric Functions}
\piccaption[]{\label{fig:unitcircle}}\parpic[r]{\begin{tikzpicture}[every node/.style={font=\small}]
 \draw[black!60,line width=0.3pt,-latex] (-2,0) -- (2,0) node[right] {$x$};
 \draw[black!60,line width=0.3pt,-latex] (0,-1.8) -- (0,2) node[above] {$y$};
 \draw [line width=1pt] (60:1.5) arc (60:360:1.5);
 \node [right] at (35:1.5) {$s=r\,\theta=\theta$};
 \draw [dashed] (0,0) -- (60:1.5) node[black,midway,above] {$1$};
 \draw [-latex,dashed] (0:0.6) arc (0:60:0.6);
 \draw [linecolor,-latex,line width=1.5pt] (0:1.5) arc (0:59:1.5);
 \fill (0:1.5) circle (2pt);
 \fill (60:1.5) circle (2pt);
 \node at (35:0.4) {$\theta$};
 \node [below right] at (0:1.5) {$1$};
 \node [below left] at (0:0) {$0$};
 \node [above right] at (60:1.5) {$(x,y)=(\cos\;\theta,\sin\;\theta)$};
 \node [right] at (-2,1.8) {$x^2 + y^2 = 1$};
\end{tikzpicture}}
The first function we will graph is the sine function. We will describe a geometrical way to
create the graph, using the \emph{unit circle}.\index{unit circle}\index{circle!unit} This is the
circle of radius $1$ in the $xy$-plane consisting of all points $(x,y)$ which satisfy the equation
$x^2 + y^2 = 1$.

We see in Figure \ref{fig:unitcircle} that any point on the unit circle has coordinates
$(x,y)=(\cos\;\theta,\sin\;\theta)$, where $\theta$ is the angle that the line segment from the
origin to $(x,y)$ makes with the positive $x$-axis (by definition of sine and cosine).
So as the point $(x,y)$ goes around the circle, its $y$-coordinate is $\sin\;\theta$.

We thus get a correspondence between the $y$-coordinates of points on the unit circle and the
values $f(\theta)=\sin\;\theta$, as shown by the horizontal lines from the unit circle to the graph
of $f(\theta)=\sin\;\theta$ in Figure \ref{fig:sinecircle} for the angles $\theta = 0$,
$\tfrac{\pi}{6}$, $\tfrac{\pi}{3}$, $\tfrac{\pi}{2}$.

\begin{figure}[ht]
 \begin{center}
  \begin{tikzpicture}[scale=1.2,every node/.style={font=\small}]
   \begin{scope}[shift={(3,0)},color=linecolor,line width=1.5pt,x=12cm/360]
	\draw[black!60,line width=0.3pt,-latex] (-5,0) -- (200,0) node[right] {$\theta$};
	\draw[black!60,line width=0.3pt,-latex] (0,-0.5) -- (0,2.4) node[above] {$f(\theta)$};
	\pgfplothandlerlineto
	\pgfplotfunction{\x}{0,5,...,180}{\pgfpointxy{\x}{2*sin(\x)}}
	\pgfusepath{stroke}
	\node[black,below left] at (0,0) {$0$};
	\foreach \pos in {30,60,90,120,150,180}
	 \draw[black!60,line width=0.3pt,shift={(\pos,0)}] (0pt,3pt) -- (0pt,-3pt);
	\foreach \pos in {2}
	 \draw[black!60,line width=0.3pt,shift={(0,\pos)}] (3pt,0pt) -- (-3pt,0pt) node[black,left]
	  {$1$};
	\node[black,below] at (30,-0.1) {$\tfrac{\pi}{6}$};
	\node[black,below] at (60,-0.1) {$\tfrac{\pi}{3}$};
	\node[black,below] at (90,-0.1) {$\tfrac{\pi}{2}$};
	\node[black,below] at (120,-0.1) {$\tfrac{2\pi}{3}$};
	\node[black,below] at (150,-0.1) {$\tfrac{5\pi}{6}$};
	\node[black,below] at (180,-0.1) {$\pi$};
	\node[black,above] at (180,1) {$f(\theta)=\sin\;\theta$};
   \end{scope}
   \draw[black!60,line width=0.3pt] (-2.2,0) -- (0,0);
   \draw[black,-latex] (0:0.8) arc (0:30:0.8);
   \draw[red,line width=0.3pt] (0,0) -- ++(30:2) node[black,pos=0.7,fill=white] {$\tfrac{\pi}{6}$}
	-- ++(2.268,0) -- ++(0,-1);
   \draw[green!60!black,line width=0.3pt] (0,0) -- ++(60:2) node[black,pos=0.7,fill=white]
	{$\tfrac{\pi}{3}$} -- ++(4,0) -- ++(0,-1.732);
   \draw[yellow!60!black,line width=0.3pt] (0,0) -- ++(90:2) node[black,pos=0.7,fill=white]
	{$\tfrac{\pi}{2}$} -- ++(6,0) -- ++(0,-2);
   \draw[cyan!80!black,line width=0.3pt] (0,0) -- (3,0) node[black,pos=0.46,fill=white] {$0$};
   \draw [black,line width=1pt] (0:2) arc (0:180:2);
   \node [black,below] at (2,0) {$1$};
   \node [black,above left] at (0,2) {$1$};
   \node [black,below] at (0,-0.2) {$x^2 + y^2 = 1$};
   \node [black] at (15:0.6) {$\theta$};
   \fill (0,0) circle (2pt);
  \end{tikzpicture}\vspace{-4mm}
 \end{center}
 \caption[]{\quad Graph of sine function based on $y$-coordinate of points on unit circle}
 \label{fig:sinecircle}
\end{figure}

We can extend the above picture to include angles from $0$ to $2\pi$ radians, as in Figure
\ref{fig:sinefullcircle}. This illustrates what is sometimes called the \emph{unit circle
definition of the sine function}.\index{sine!unit circle definition of}
\newpage
\begin{figure}[ht]
 \begin{center}
  \begin{tikzpicture}[scale=1.2,every node/.style={font=\small}]
   \begin{scope}[shift={(3,0)},color=linecolor,line width=1.5pt,x=6cm/360]
	\draw[black!60,line width=0.3pt,-latex] (-5,0) -- (380,0) node[right] {$\theta$};
	\draw[black!60,line width=0.3pt,-latex] (0,-1.2) -- (0,1.3) node[above] {$f(\theta)$};
	\pgfplothandlerlineto
	\pgfplotfunction{\x}{0,5,...,360}{\pgfpointxy{\x}{sin(\x)}}
	\pgfusepath{stroke}
	\node[black,below left] at (0,0) {$0$};
	\foreach \pos in {30,60,90,120,150,180,225,270,315,360}
	 \draw[black!60,line width=0.3pt,shift={(\pos,0)}] (0pt,3pt) -- (0pt,-3pt);
	\foreach \pos in {-1,1}
	 \draw[black!60,line width=0.3pt,shift={(0,\pos)}] (3pt,0pt) -- (-3pt,0pt) node[black,left]
	  {$\pos$};
	\node[black,below] at (30,-0.1) {$\tfrac{\pi}{6}$};
	\node[black,below] at (60,-0.1) {$\tfrac{\pi}{3}$};
	\node[black,below] at (90,-0.1) {$\tfrac{\pi}{2}$};
	\node[black,below] at (120,-0.1) {$\tfrac{2\pi}{3}$};
	\node[black,below] at (150,-0.1) {$\tfrac{5\pi}{6}$};
	\node[black,below] at (180,-0.1) {$\pi$};
	\node[black,below] at (225,-0.1) {$\tfrac{5\pi}{4}$};
	\node[black,below] at (270,-0.1) {$\tfrac{3\pi}{2}$};
	\node[black,below] at (315,-0.1) {$\tfrac{7\pi}{4}$};
	\node[black,below] at (360,-0.1) {$2\pi$};
	\node[black,above] at (180,1) {$f(\theta)=\sin\;\theta$};
   \end{scope}
   \draw[black!60,line width=0.3pt,-latex] (-1.5,0) -- (1.5,0) node[below right] {$x$};
   \draw[black!60,line width=0.3pt,-latex] (0,-1) -- (0,1.3) node[above] {$y$};
   \draw[black,-latex] (0:0.8) arc (0:30:0.8);
   \draw[red,line width=0.3pt] (0,0) -- ++(30:1);
   \draw[red,line width=0.3pt] (0,0) -- ++(150:1) -- ++(4.366,0) -- ++(0,-0.5);
   \draw[red,line width=0.3pt] (3.5,0.5) -- (5.5,0.5) -- (5.5,0);
   \draw[green!60!black,line width=0.3pt] (0,0) -- ++(60:1);
   \draw[green!60!black,line width=0.3pt] (0,0) -- ++(120:1) -- ++(4.5,0) -- ++(0,-0.866);
   \draw[green!60!black,line width=0.3pt] (4,0.866) -- (5,0.866) -- (5,0);
   \draw[yellow!60!black,line width=0.3pt] (0,0) -- ++(90:1) -- ++(4.5,0) -- ++(0,-1);
   \draw[cyan!80!black,line width=0.3pt] (0,0) -- (3,0);
   \draw[blue,line width=0.3pt] (0,0) -- (225:1) -- ++(7.457,0) -- ++(0,0.707);
   \draw[blue,line width=0.3pt] (0,0) -- (315:1) -- ++(7.543,0) -- ++(0,0.707);
   \draw [black,line width=1pt] (0,0) circle (1);
   \node [black,below right] at (1,0) {$1$};
   \node [black,below] at (0,-1) {$x^2 + y^2 = 1$};
   \node [black] at (15:0.6) {$\theta$};
  \end{tikzpicture}\vspace{-6mm}
 \end{center}
 \caption[]{\quad Unit circle definition of the sine function}
 \label{fig:sinefullcircle}
\end{figure}

Since the trigonometric functions repeat every $2\pi$ radians ($360\Degrees$), we get, for example,
the following graph of the function $y=\sin\;x$ for $x$ in the interval $\ival{-2\pi}{2\pi}$:

\begin{figure}[ht]
 \begin{center}
  \begin{tikzpicture}[scale=1.1,every node/.style={font=\small}]
   \begin{scope}[shift={(3,0)},color=linecolor,line width=1.5pt,x=6cm/360]
    \draw[black!60,line width=0.3pt,dotted] (-360,-1) grid[xstep=45,ystep=1] (360,1);
	\draw[black!60,line width=0.3pt,-latex] (-380,0) -- (380,0) node[right] {$x$};
	\draw[black!60,line width=0.3pt,-latex] (0,-1.2) -- (0,1.5) node[above] {$y$};
	\pgfplothandlerlineto
	\pgfplotfunction{\x}{-360,-355,...,360}{\pgfpointxy{\x}{sin(\x)}}
	\pgfusepath{stroke}
	\node[black,below right] at (0,0) {$0$};
	\foreach \pos in {-360,-315,-270,-225,-180,-135,-90,-45,45,90,135,180,225,270,315,360}
	 \draw[black!60,line width=0.3pt,shift={(\pos,0)}] (0pt,3pt) -- (0pt,-3pt);
	\foreach \pos in {-1,1}
	 \draw[black!60,line width=0.3pt,shift={(0,\pos)}] (3pt,0pt) -- (-3pt,0pt) node[black,left]
      {$\pos$};
	\node[black,below] at (45,-0.1) {$\tfrac{\pi}{4}$};
	\node[black,below] at (90,-0.1) {$\tfrac{\pi}{2}$};
	\node[black,below] at (135,-0.1) {$\tfrac{3\pi}{4}$};
	\node[black,below] at (180,-0.1) {$\pi$};
	\node[black,below] at (225,-0.1) {$\tfrac{5\pi}{4}$};
	\node[black,below] at (270,-0.1) {$\tfrac{3\pi}{2}$};
	\node[black,below] at (315,-0.1) {$\tfrac{7\pi}{4}$};
	\node[black,below] at (360,-0.1) {$2\pi$};
	\node[black,below] at (-45,-0.1) {$-\tfrac{\pi}{4}$};
	\node[black,below] at (-90,-0.1) {$-\tfrac{\pi}{2}$};
	\node[black,below] at (-135,-0.1) {$-\tfrac{3\pi}{4}$};
	\node[black,below] at (-180,-0.1) {$-\pi$};
	\node[black,below] at (-225,-0.1) {$-\tfrac{5\pi}{4}$};
	\node[black,below] at (-270,-0.1) {$-\tfrac{3\pi}{2}$};
	\node[black,below] at (-315,-0.1) {$-\tfrac{7\pi}{4}$};
	\node[black,below] at (-360,-0.1) {$-2\pi$};
	\node[black,above] at (180,1) {$y=\sin\;x$};
   \end{scope}
  \end{tikzpicture}\vspace{-6mm}
 \end{center}
 \caption[]{\quad Graph of $y=\sin\;x$}
 \label{fig:sinegraph}
\end{figure}

To graph the cosine function, we could again use the unit circle idea (using the $x$-coordinate of
a point that moves around the circle), but there is an easier way. Recall from Section 1.5 that
$\cos\;x = \sin\;(x+90\Degrees)$ for all $x$. So $\cos\;0\Degrees$ has the same value as
$\sin\;90\Degrees$, $\cos\;90\Degrees$ has the same value as $\sin\;180\Degrees$,
$\cos\;180\Degrees$ has the same value as $\sin\;270\Degrees$,
and so on. In other words, the graph of the cosine function is just the
graph of the sine function shifted to the \emph{left} by $90\Degrees = \pi/2$ radians,
as in Figure \ref{fig:cosinegraph}:\index{sine!graph of}\index{cosine!graph of}

\begin{figure}[ht]
 \begin{center}
  \begin{tikzpicture}[scale=1.1,every node/.style={font=\small}]
   \begin{scope}[shift={(3,0)},color=linecolor,line width=1.5pt,x=6cm/360]
    \draw[black!60,line width=0.3pt,dotted] (-360,-1) grid[xstep=45,ystep=1] (360,1);
	\draw[black!60,line width=0.3pt,-latex] (-380,0) -- (380,0) node[right] {$x$};
	\draw[black!60,line width=0.3pt,-latex] (0,-1.2) -- (0,1.5) node[above] {$y$};
	\pgfplothandlerlineto
	\pgfplotfunction{\x}{-360,-355,...,360}{\pgfpointxy{\x}{cos(\x)}}
	\pgfusepath{stroke}
	\node[black,below left] at (0,0) {$0$};
	\foreach \pos in {-360,-315,-270,-225,-180,-135,-90,-45,45,90,135,180,225,270,315,360}
	 \draw[black!60,line width=0.3pt,shift={(\pos,0)}] (0pt,3pt) -- (0pt,-3pt);
	\foreach \pos in {-1,1}
	 \draw[black!60,line width=0.3pt,shift={(0,\pos)}] (3pt,0pt) -- (-3pt,0pt) node[black,left]
      {$\pos$};
	\node[black,below] at (45,-0.1) {$\tfrac{\pi}{4}$};
	\node[black,below] at (90,-0.1) {$\tfrac{\pi}{2}$};
	\node[black,below] at (135,-0.1) {$\tfrac{3\pi}{4}$};
	\node[black,below] at (180,-0.1) {$\pi$};
	\node[black,below] at (225,-0.1) {$\tfrac{5\pi}{4}$};
	\node[black,below] at (270,-0.1) {$\tfrac{3\pi}{2}$};
	\node[black,below] at (315,-0.1) {$\tfrac{7\pi}{4}$};
	\node[black,below] at (360,-0.1) {$2\pi$};
	\node[black,below] at (-45,-0.1) {$-\tfrac{\pi}{4}$};
	\node[black,below] at (-90,-0.1) {$-\tfrac{\pi}{2}$};
	\node[black,below] at (-135,-0.1) {$-\tfrac{3\pi}{4}$};
	\node[black,below] at (-180,-0.1) {$-\pi$};
	\node[black,below] at (-225,-0.1) {$-\tfrac{5\pi}{4}$};
	\node[black,below] at (-270,-0.1) {$-\tfrac{3\pi}{2}$};
	\node[black,below] at (-315,-0.1) {$-\tfrac{7\pi}{4}$};
	\node[black,below] at (-360,-0.1) {$-2\pi$};
	\node[black,above] at (180,1) {$y=\cos\;x$};
   \end{scope}
  \end{tikzpicture}\vspace{-6mm}
 \end{center}
 \caption[]{\quad Graph of $y=\cos\;x$}
 \label{fig:cosinegraph}
\end{figure}

To graph the tangent function, use $\tan\;x = \frac{\sin\;x}{\cos\;x}$ to get the following graph:
\newpage
\begin{figure}[ht]
 \begin{center}
  \begin{tikzpicture}[scale=1.1,every node/.style={font=\small}]
   \begin{scope}[shift={(3,0)},color=linecolor,line width=1.5pt,x=6cm/360,y=4cm/10]
    \draw[black!60,line width=0.3pt,dotted] (-360,-10) grid[xstep=45,ystep=2] (360,10);
	\draw[black!60,line width=0.3pt,-latex] (-380,0) -- (380,0) node[right] {$x$};
	\draw[black!60,line width=0.3pt,-latex] (0,-10) -- (0,10) node[above] {$y$};
	\draw[linecolor,line width=0.5pt,dashed] (90,-10) -- (90,10);
	\draw[linecolor,line width=0.5pt,dashed] (270,-10) -- (270,10);
	\draw[linecolor,line width=0.5pt,dashed] (-90,-10) -- (-90,10);
	\draw[linecolor,line width=0.5pt,dashed] (-270,-10) -- (-270,10);
	\pgfplothandlerlineto
	\pgfplotfunction{\x}{-360,-356,...,-276}{\pgfpointxy{\x}{tan(\x)}}
	\pgfplotfunction{\x}{-264,-260,...,-96}{\pgfpointxy{\x}{tan(\x)}}
	\pgfplotfunction{\x}{-84,-80,...,84}{\pgfpointxy{\x}{tan(\x)}}
	\pgfplotfunction{\x}{96,100,...,264}{\pgfpointxy{\x}{tan(\x)}}
	\pgfplotfunction{\x}{276,280,...,360}{\pgfpointxy{\x}{tan(\x)}}
	\pgfusepath{stroke}
	\node[black,below right] at (0,0) {$0$};
	\foreach \pos in {-360,-315,-270,-225,-180,-135,-90,-45,45,90,135,180,225,270,315,360}
	 \draw[black!60,line width=0.3pt,shift={(\pos,0)}] (0pt,3pt) -- (0pt,-3pt);
	\foreach \pos in {-8,-6,-4,-2,2,4,6,8}
	 \draw[black!60,line width=0.3pt,shift={(0,\pos)}] (3pt,0pt) -- (-3pt,0pt) node[black,left]
      {$\pos$};
	\node[black,below] at (45,-0.1) {$\tfrac{\pi}{4}$};
	\node[black,below] at (90,-0.1) {$\tfrac{\pi}{2}$};
	\node[black,below] at (135,-0.1) {$\tfrac{3\pi}{4}$};
	\node[black,below] at (180,-0.1) {$\pi$};
	\node[black,below] at (225,-0.1) {$\tfrac{5\pi}{4}$};
	\node[black,below] at (270,-0.1) {$\tfrac{3\pi}{2}$};
	\node[black,below] at (315,-0.1) {$\tfrac{7\pi}{4}$};
	\node[black,below] at (360,-0.1) {$2\pi$};
	\node[black,below] at (-45,-0.1) {$-\tfrac{\pi}{4}$};
	\node[black,below] at (-90,-0.1) {$-\tfrac{\pi}{2}$};
	\node[black,below] at (-135,-0.1) {$-\tfrac{3\pi}{4}$};
	\node[black,below] at (-180,-0.1) {$-\pi$};
	\node[black,below] at (-225,-0.1) {$-\tfrac{5\pi}{4}$};
	\node[black,below] at (-270,-0.1) {$-\tfrac{3\pi}{2}$};
	\node[black,below] at (-315,-0.1) {$-\tfrac{7\pi}{4}$};
	\node[black,below] at (-360,-0.1) {$-2\pi$};
	\node[black,above] at (180,1) {$y=\tan\;x$};
   \end{scope}
  \end{tikzpicture}\vspace{-6mm}
 \end{center}
 \caption[]{\quad Graph of $y=\tan\;x$}
 \label{fig:tangentgraph}
\end{figure}

Recall that the tangent is positive for angles in QI and QIII, and is negative in QII and QIV, and
that is indeed what the graph in Figure \ref{fig:tangentgraph} shows. We know that $\tan\;x$ is
not defined when $\cos\;x = 0$, i.e. at odd multiples of $\frac{\pi}{2}$: $x=\pm\,\frac{\pi}{2}$,
$\pm\,\frac{3\pi}{2}$, $\pm\,\frac{5\pi}{2}$, etc. We can figure out what happens \emph{near} those
angles by looking at the sine and cosine functions. For example, for $x$ in QI near $\frac{\pi}{2}$,
$\sin\;x$ and $\cos\;x$ are both positive, with $\sin\;x$ very close to $1$ and $\cos\;x$ very close
to $0$, so the quotient $\tan\;x = \frac{\sin\;x}{\cos\;x}$ is a positive number that is very large.
And the closer $x$ gets to $\frac{\pi}{2}$, the larger $\tan\;x$ gets. Thus, $x=\frac{\pi}{2}$ is a
\emph{vertical asymptote}\index{vertical asymptote}\index{asymptote!vertical} of the graph of
$y=\tan\;x$.\index{tangent!graph of}\index{asymptote}

Likewise, for $x$ in QII very close to $\frac{\pi}{2}$, $\sin\;x$ is very close to $1$ and $\cos\;x$
is negative and very close to $0$, so the quotient $\tan\;x = \frac{\sin\;x}{\cos\;x}$ is a negative
number that is very large, and it gets larger in the negative direction the closer $x$ gets to
$\frac{\pi}{2}$. The graph shows this. Similarly, we get vertical asymptotes at $x=-\frac{\pi}{2}$,
$x=\frac{3\pi}{2}$, and $x=-\frac{3\pi}{2}$, as in Figure \ref{fig:tangentgraph}. Notice that
the graph of the tangent function repeats every $\pi$ radians, i.e. two times faster than the
graphs of sine and cosine repeat.

The graphs of the remaining trigonometric functions can be determined by looking at the graphs of
their reciprocal functions. For example, using $\csc\;x = \frac{1}{\sin\;x}$ we can just look at
the graph of $y=\sin\;x$ and invert the values. We will get vertical asymptotes when $\sin\;x=0$,
namely at multiples of $\pi$: $x=0$, $\pm\,\pi$, $\pm\,2\pi$, etc. Figure \ref{fig:cosecantgraph}
shows the graph of $y=\csc\;x$, with the graph of $y=\sin\;x$ (the dashed curve) for reference.
\newpage
\begin{figure}[ht]
 \begin{center}
  \begin{tikzpicture}[scale=1.1,every node/.style={font=\small}]
   \begin{scope}[shift={(3,0)},dashed,x=6cm/360,y=3cm/4]
    \draw[black!60,line width=0.3pt,dotted] (-360,-4) grid[xstep=45,ystep=1] (360,4);
	\draw[black!60,solid,line width=0.3pt,-latex] (-380,0) -- (380,0) node[right] {$x$};
	\draw[black!60,solid,line width=0.3pt,-latex] (0,-4.2) -- (0,4.5) node[above] {$y$};
	\pgfplothandlerlineto
	\pgfplotfunction{\x}{-360,-355,...,360}{\pgfpointxy{\x}{sin(\x)}}
	\pgfusepath{stroke}
	\node[black,below right] at (0,0) {$0$};
	\foreach \pos in {-360,-315,-270,-225,-180,-135,-90,-45,45,90,135,180,225,270,315,360}
	 \draw[black!60,solid,line width=0.3pt,shift={(\pos,0)}] (0pt,3pt) -- (0pt,-3pt);
	\foreach \pos in {-4,-3,-2,-1,1,2,3,4}
	 \draw[black!60,solid,line width=0.3pt,shift={(0,\pos)}] (3pt,0pt) -- (-3pt,0pt)
	  node[black,left] {$\pos$};
	\node[black,below] at (45,-0.1) {$\tfrac{\pi}{4}$};
	\node[black,below] at (90,-0.1) {$\tfrac{\pi}{2}$};
	\node[black,below] at (135,-0.1) {$\tfrac{3\pi}{4}$};
	\node[black,below] at (180,-0.1) {$\pi$};
	\node[black,below] at (225,-0.1) {$\tfrac{5\pi}{4}$};
	\node[black,below] at (270,-0.1) {$\tfrac{3\pi}{2}$};
	\node[black,below] at (315,-0.1) {$\tfrac{7\pi}{4}$};
	\node[black,below] at (360,-0.1) {$2\pi$};
	\node[black,below] at (-45,-0.1) {$-\tfrac{\pi}{4}$};
	\node[black,below] at (-90,-0.1) {$-\tfrac{\pi}{2}$};
	\node[black,below] at (-135,-0.1) {$-\tfrac{3\pi}{4}$};
	\node[black,below] at (-180,-0.1) {$-\pi$};
	\node[black,below] at (-225,-0.1) {$-\tfrac{5\pi}{4}$};
	\node[black,below] at (-270,-0.1) {$-\tfrac{3\pi}{2}$};
	\node[black,below] at (-315,-0.1) {$-\tfrac{7\pi}{4}$};
	\node[black,below] at (-360,-0.1) {$-2\pi$};
	\node[black,above] at (90,3) {$y=\csc\;x$};
   \end{scope}
   \begin{scope}[shift={(3,0)},color=linecolor,line width=1.5pt,x=6cm/360,y=3cm/4]
	\draw[line width=0.5pt,dashed] (360,-4) -- (360,4);
	\draw[line width=0.5pt,dashed] (180,-4) -- (180,4);
	\draw[line width=0.5pt,dashed] (-360,-4) -- (-360,4);
	\draw[line width=0.5pt,dashed] (-180,-4) -- (-180,4);
	\pgfplothandlerlineto
	\pgfplotfunction{\x}{-345,-340,...,-195}{\pgfpointxy{\x}{1/sin(\x)}}
	\pgfplotfunction{\x}{-165,-160,...,-15}{\pgfpointxy{\x}{1/sin(\x)}}
	\pgfplotfunction{\x}{15,20,...,165}{\pgfpointxy{\x}{1/sin(\x)}}
	\pgfplotfunction{\x}{195,200,...,345}{\pgfpointxy{\x}{1/sin(\x)}}
	\pgfusepath{stroke}
   \end{scope}
  \end{tikzpicture}\vspace{-6mm}
 \end{center}
 \caption[]{\quad Graph of $y=\csc\;x$}
 \label{fig:cosecantgraph}
\end{figure}

Likewise, Figure \ref{fig:secantgraph} shows the graph of $y=\sec\;x$, with the graph of
$y=\cos\;x$ (the dashed curve) for reference. Note the vertical asymptotes at\index{secant!graph of}
$x=\pm\,\frac{\pi}{2}$, $\pm\,\frac{3\pi}{2}$. Notice also that the graph is just the graph of
the cosecant function shifted to the left by $\frac{\pi}{2}$ radians.\index{cosecant!graph of}

\begin{figure}[ht]
 \begin{center}
  \begin{tikzpicture}[scale=1.1,every node/.style={font=\small}]
   \begin{scope}[shift={(3,0)},dashed,x=6cm/360,y=3cm/4]
    \draw[black!60,line width=0.3pt,dotted] (-360,-4) grid[xstep=45,ystep=1] (360,4);
	\draw[black!60,solid,line width=0.3pt,-latex] (-380,0) -- (380,0) node[right] {$x$};
	\draw[black!60,solid,line width=0.3pt,-latex] (0,-4.2) -- (0,4.5) node[above] {$y$};
	\pgfplothandlerlineto
	\pgfplotfunction{\x}{-360,-355,...,360}{\pgfpointxy{\x}{cos(\x)}}
	\pgfusepath{stroke}
	\node[black,below right] at (0,0) {$0$};
	\foreach \pos in {-360,-315,-270,-225,-180,-135,-90,-45,45,90,135,180,225,270,315,360}
	 \draw[black!60,solid,line width=0.3pt,shift={(\pos,0)}] (0pt,3pt) -- (0pt,-3pt);
	\foreach \pos in {-4,-3,-2,-1,1,2,3,4}
	 \draw[black!60,solid,line width=0.3pt,shift={(0,\pos)}] (3pt,0pt) -- (-3pt,0pt)
	  node[black,left] {$\pos$};
	\node[black,below] at (45,-0.1) {$\tfrac{\pi}{4}$};
	\node[black,below] at (90,-0.1) {$\tfrac{\pi}{2}$};
	\node[black,below] at (135,-0.1) {$\tfrac{3\pi}{4}$};
	\node[black,below] at (180,-0.1) {$\pi$};
	\node[black,below] at (225,-0.1) {$\tfrac{5\pi}{4}$};
	\node[black,below] at (270,-0.1) {$\tfrac{3\pi}{2}$};
	\node[black,below] at (315,-0.1) {$\tfrac{7\pi}{4}$};
	\node[black,below] at (360,-0.1) {$2\pi$};
	\node[black,below] at (-45,-0.1) {$-\tfrac{\pi}{4}$};
	\node[black,below] at (-90,-0.1) {$-\tfrac{\pi}{2}$};
	\node[black,below] at (-135,-0.1) {$-\tfrac{3\pi}{4}$};
	\node[black,below] at (-180,-0.1) {$-\pi$};
	\node[black,below] at (-225,-0.1) {$-\tfrac{5\pi}{4}$};
	\node[black,below] at (-270,-0.1) {$-\tfrac{3\pi}{2}$};
	\node[black,below] at (-315,-0.1) {$-\tfrac{7\pi}{4}$};
	\node[black,below] at (-360,-0.1) {$-2\pi$};
	\node[black,above] at (180,3) {$y=\sec\;x$};
   \end{scope}
   \begin{scope}[shift={(3,0)},color=linecolor,line width=1.5pt,x=6cm/360,y=3cm/4]
	\draw[line width=0.5pt,dashed] (270,-4) -- (270,4);
	\draw[line width=0.5pt,dashed] (90,-4) -- (90,4);
	\draw[line width=0.5pt,dashed] (-270,-4) -- (-270,4);
	\draw[line width=0.5pt,dashed] (-90,-4) -- (-90,4);
	\pgfplothandlerlineto
	\pgfplotfunction{\x}{-360,-355,...,-285}{\pgfpointxy{\x}{1/cos(\x)}}
	\pgfplotfunction{\x}{-255,-250,...,-105}{\pgfpointxy{\x}{1/cos(\x)}}
	\pgfplotfunction{\x}{-75,-70,...,75}{\pgfpointxy{\x}{1/cos(\x)}}
	\pgfplotfunction{\x}{105,110,...,255}{\pgfpointxy{\x}{1/cos(\x)}}
	\pgfplotfunction{\x}{285,290,...,360}{\pgfpointxy{\x}{1/cos(\x)}}
	\pgfusepath{stroke}
   \end{scope}
  \end{tikzpicture}\vspace{-6mm}
 \end{center}
 \caption[]{\quad Graph of $y=\sec\;x$}
 \label{fig:secantgraph}
\end{figure}
\newpage
The graph of $y=\cot\;x$ can also be determined by using $\cot\;x = \frac{1}{\cot\;x}$.
Alternatively, we can\index{cotangent!graph of}
use the relation $\cot\;x = -\tan\;(x+90\Degrees)$ from Section 1.5, so that the graph of the
cotangent function is just the graph of the tangent function shifted to the left by $\frac{\pi}{2}$
radians and then reflected about the $x$-axis, as in Figure \ref{fig:cotangentgraph}:

\begin{figure}[ht]
 \begin{center}
  \begin{tikzpicture}[scale=1.1,every node/.style={font=\small}]
   \begin{scope}[shift={(3,0)},color=linecolor,line width=1.5pt,x=6cm/360,y=4cm/10]
    \draw[black!60,line width=0.3pt,dotted] (-360,-10) grid[xstep=45,ystep=2] (360,10);
	\draw[black!60,line width=0.3pt,-latex] (-380,0) -- (380,0) node[right] {$x$};
	\draw[black!60,line width=0.3pt,-latex] (0,-10) -- (0,10) node[above] {$y$};
	\draw[linecolor,line width=0.5pt,dashed] (180,-10) -- (180,10);
	\draw[linecolor,line width=0.5pt,dashed] (360,-10) -- (360,10);
	\draw[linecolor,line width=0.5pt,dashed] (-180,-10) -- (-180,10);
	\draw[linecolor,line width=0.5pt,dashed] (-360,-10) -- (-360,10);
	\pgfplothandlerlineto
	\pgfplotfunction{\x}{-354,-348,...,-186}{\pgfpointxy{\x}{-tan(90+\x)}}
	\pgfplotfunction{\x}{-174,-168,...,-6}{\pgfpointxy{\x}{-tan(90+\x)}}
	\pgfplotfunction{\x}{6,12,...,174}{\pgfpointxy{\x}{-tan(90+\x)}}
	\pgfplotfunction{\x}{186,192,...,354}{\pgfpointxy{\x}{-tan(90+\x)}}
	\pgfusepath{stroke}
	\node[black,below right] at (0,0) {$0$};
	\foreach \pos in {-360,-315,-270,-225,-180,-135,-90,-45,45,90,135,180,225,270,315,360}
	 \draw[black!60,line width=0.3pt,shift={(\pos,0)}] (0pt,3pt) -- (0pt,-3pt);
	\foreach \pos in {-8,-6,-4,-2,2,4,6,8}
	 \draw[black!60,line width=0.3pt,shift={(0,\pos)}] (3pt,0pt) -- (-3pt,0pt) node[black,left]
      {$\pos$};
	\node[black,below] at (45,-0.1) {$\tfrac{\pi}{4}$};
	\node[black,below] at (90,-0.1) {$\tfrac{\pi}{2}$};
	\node[black,below] at (135,-0.1) {$\tfrac{3\pi}{4}$};
	\node[black,below] at (180,-0.1) {$\pi$};
	\node[black,below] at (225,-0.1) {$\tfrac{5\pi}{4}$};
	\node[black,below] at (270,-0.1) {$\tfrac{3\pi}{2}$};
	\node[black,below] at (315,-0.1) {$\tfrac{7\pi}{4}$};
	\node[black,below] at (360,-0.1) {$2\pi$};
	\node[black,below] at (-45,-0.1) {$-\tfrac{\pi}{4}$};
	\node[black,below] at (-90,-0.1) {$-\tfrac{\pi}{2}$};
	\node[black,below] at (-135,-0.1) {$-\tfrac{3\pi}{4}$};
	\node[black,below] at (-180,-0.1) {$-\pi$};
	\node[black,below] at (-225,-0.1) {$-\tfrac{5\pi}{4}$};
	\node[black,below] at (-270,-0.1) {$-\tfrac{3\pi}{2}$};
	\node[black,below] at (-315,-0.1) {$-\tfrac{7\pi}{4}$};
	\node[black,below] at (-360,-0.1) {$-2\pi$};
	\node[black,above] at (90,1) {$y=\cot\;x$};
   \end{scope}
  \end{tikzpicture}\vspace{-6mm}
 \end{center}
 \caption[]{\quad Graph of $y=\cot\;x$}
 \label{fig:cotangentgraph}
\end{figure}

\begin{exmp}
 Draw the graph of $y=-\sin\;x$ for $0 \le x \le 2\pi$.\vspace{1mm}
 \par\noindent\textbf{Solution:} Multiplying a function by $-1$ just reflects its graph around the
 $x$-axis. So reflecting the graph of $y=\sin\;x$ around the $x$-axis gives us the graph of
 $y=-\sin\;x$:

 \begin{center}
  \begin{tikzpicture}[scale=1.2,every node/.style={font=\small}]
   \begin{scope}[shift={(3,0)},color=linecolor,line width=1.5pt,x=6cm/360]
    \draw[black!60,line width=0.3pt,dotted] (0,-1) grid[xstep=45,ystep=1] (360,1);
	\draw[black!60,line width=0.3pt,-latex] (0,0) -- (380,0) node[right] {$x$};
	\draw[black!60,line width=0.3pt,-latex] (0,-1.2) -- (0,1.5) node[above] {$y$};
	\pgfplothandlerlineto
	\pgfplotfunction{\x}{0,5,...,360}{\pgfpointxy{\x}{-sin(\x)}}
	\pgfusepath{stroke}
	\node[black,left] at (0,0) {$0$};
	\foreach \pos in {45,90,135,180,225,270,315,360}
	 \draw[black!60,line width=0.3pt,shift={(\pos,0)}] (0pt,3pt) -- (0pt,-3pt);
	\foreach \pos in {-1,1}
	 \draw[black!60,line width=0.3pt,shift={(0,\pos)}] (3pt,0pt) -- (-3pt,0pt) node[black,left]
      {$\pos$};
	\node[black,below] at (45,-0.1) {$\tfrac{\pi}{4}$};
	\node[black,below] at (90,-0.1) {$\tfrac{\pi}{2}$};
	\node[black,below] at (135,-0.1) {$\tfrac{3\pi}{4}$};
	\node[black,below] at (180,-0.1) {$\pi$};
	\node[black,below] at (225,-0.1) {$\tfrac{5\pi}{4}$};
	\node[black,below] at (270,-0.1) {$\tfrac{3\pi}{2}$};
	\node[black,below] at (315,-0.1) {$\tfrac{7\pi}{4}$};
	\node[black,below] at (360,-0.1) {$2\pi$};
	\node[black,above] at (180,1) {$y=-\sin\;x$};
   \end{scope}
  \end{tikzpicture}
 \end{center}

\noindent Note that this graph is the same as the graphs of $y=\sin\;(x \pm \pi)$ and
 $y=\cos\;(x+\frac{\pi}{2})$.
\end{exmp}
\divider
\newpage
It is worthwhile to remember the general shapes of the graphs of the six trigonometric functions,
especially for sine, cosine, and tangent. In particular, the graphs of the sine and cosine functions
are called \emph{sinusoidal}\index{sinusoidal curves} curves. Many phenomena in nature exhibit
sinusoidal behavior, so recognizing the general shape is important.

\begin{exmp}
 Draw the graph of $y=1+\cos\;x$ for $0 \le x \le 2\pi$.\vspace{1mm}
 \par\noindent\textbf{Solution:} Adding a constant to a function just moves its graph up or down by
 that amount, depending on whether the constant is positive or negative, respectively. So adding
 $1$ to $\cos\;x$ moves the graph of $y=\cos\;x$ upward by $1$, giving us the graph of
 $y=1+\cos\;x$:

 \begin{center}
  \begin{tikzpicture}[scale=1.2,every node/.style={font=\small}]
   \begin{scope}[shift={(3,0)},color=linecolor,line width=1.5pt,x=6cm/360]
    \draw[black!60,line width=0.3pt,dotted] (0,0) grid[xstep=45,ystep=1] (360,2);
	\draw[black!60,line width=0.3pt,-latex] (0,0) -- (380,0) node[right] {$x$};
	\draw[black!60,line width=0.3pt,-latex] (0,0) -- (0,2.5) node[above] {$y$};
	\pgfplothandlerlineto
	\pgfplotfunction{\x}{0,5,...,360}{\pgfpointxy{\x}{1+cos(\x)}}
	\pgfusepath{stroke}
	\node[black,below left] at (0,0) {$0$};
	\foreach \pos in {45,90,135,180,225,270,315,360}
	 \draw[black!60,line width=0.3pt,shift={(\pos,0)}] (0pt,3pt) -- (0pt,-3pt);
	\foreach \pos in {1,2}
	 \draw[black!60,line width=0.3pt,shift={(0,\pos)}] (3pt,0pt) -- (-3pt,0pt) node[black,left]
      {$\pos$};
	\node[black,below] at (45,-0.1) {$\tfrac{\pi}{4}$};
	\node[black,below] at (90,-0.1) {$\tfrac{\pi}{2}$};
	\node[black,below] at (135,-0.1) {$\tfrac{3\pi}{4}$};
	\node[black,below] at (180,-0.1) {$\pi$};
	\node[black,below] at (225,-0.1) {$\tfrac{5\pi}{4}$};
	\node[black,below] at (270,-0.1) {$\tfrac{3\pi}{2}$};
	\node[black,below] at (315,-0.1) {$\tfrac{7\pi}{4}$};
	\node[black,below] at (360,-0.1) {$2\pi$};
	\node[black,above] at (180,1) {$y=1+\cos\;x$};
   \end{scope}
  \end{tikzpicture}
 \end{center}
\end{exmp}\vspace{-4mm}
\divider
\vspace{2mm}

\startexercises\label{sec5dot1}
\vspace{4mm}
{\small
\par\noindent For Exercises 1-12, draw the graph of the given function for $0 \le x \le 2\pi$.
\begin{enumerate}[\bfseries 1.]
\begin{multicols}{4}
 \item $y=-\cos\;x$
 \item $y=1+\sin\;x$
 \item $y=2-\cos\;x$
 \item $y=2-\sin\;x$
\end{multicols}
\begin{multicols}{4}
 \item $y=-\tan\;x$
 \item $y=-\cot\;x$
 \item $y=1+\sec\;x$
 \item $y=-1-\csc\;x$
\end{multicols}
\begin{multicols}{4}
 \item $y=2\sin\;x$
 \item $y=-3\cos\;x$
 \item $y=-2\tan\;x$
 \item $y=-2\sec\;x$
\end{multicols}
\piccaption[]{\label{fig:linedef}}\parpic[r]{\begin{tikzpicture}[every node/.style={font=\small},
scale=0.8]
 \draw[black!60,line width=0.3pt,-latex] (-2.2,0) -- (2.8,0) node[right] {$x$};
 \draw[black!60,line width=0.3pt,-latex] (0,-2.2) -- (0,2.8) node[above] {$y$};
 \draw [linecolor,line width=1.5pt] (0,0) circle (2);
 \draw [line width=1pt] (0,0) -- (2,3.464) -- (2,0);
 \draw [line width=1pt] (60:2) -- (1,0);
 \draw [line width=1pt] (0,2) -- (1.155,2);
 \fill (0,0) circle (2pt);
 \node [below left] at (0,0) {$O$};
 \node [below] at (1,0) {$M$};
 \node [above right] at (2,0) {$N$};
 \node [left] at (60:1.9) {$P$};
 \node [right] at (2,3.464) {$Q$};
 \node [above left] at (0.1,2) {$R$};
 \node [above] at (1.1,2) {$S$};
 \node [below right] at (2,0) {$1$};
 \node at (0.4,0.2) {$\theta$};
 \draw [dashed] (0:0.7) arc (0:60:0.7);
\end{tikzpicture}}
 \item\label{exmp:linedef}
  We can extend the unit circle definition of the sine and cosine functions to all six
  trigonometric functions. Let $P$ be a point in QI on the unit circle, so that the line segment
  $\overline{OP}$ in Figure \ref{fig:linedef} has length $1$ and makes an acute angle $\theta$
  with the positive $x$-axis. Identify each of the six trigonometric functions of $\theta$ with
  exactly one of the line segments in Figure \ref{fig:linedef}, keeping in mind that the radius of
  the circle is $1$. To get you started, we have $\sin\;\theta = MP$ (why?).
 \item For Exercise \ref{exmp:linedef},
  how would you draw the line segments in Figure \ref{fig:linedef} if $\theta$ was in QII?
  Recall that some of the trigonometric functions are negative in QII, so you will have to come up
  with a convention for how to treat some of the line segment lengths as negative.
 \item For any point $(x,y)$ on the unit circle and any angle $\alpha$, show that the point
  $R_{\alpha} (x,y)$ defined by $R_{\alpha} (x,y)
  = (x\,\cos\;\alpha \,-\, y\,\sin\;\alpha , x\,\sin\;\alpha \,+\, y\,\cos\;\alpha)$ is also on the
  unit circle. What is the geometric interpretation of $R_{\alpha} (x,y)$? Also, show that
  $R_{-\alpha} (R_{\alpha} (x,y)) = (x,y)$ and $R_{\beta} (R_{\alpha} (x,y)) = R_{\alpha + \beta}
  (x,y)$.
\end{enumerate}}

\newpage
%Begin Section 5.2
\section{Properties of Graphs of Trigonometric Functions}
We saw in Section 5.1 how the graphs of the trigonometric functions repeat every $2\pi$ radians.
In this section we will discuss this and other properties of graphs, especially for the sinusoidal
functions (sine and cosine).

First, recall that the \textbf{domain}\index{domain} of a function $f(x)$ is the
set of all numbers $x$ for which the function is defined. For example, the domain of $f(x) =
\sin\;x$ is the set of all real numbers, whereas the domain of $f(x) = \tan\;x$ is the set of all
real numbers except $x=\pm\,\frac{\pi}{2}$, $\pm\,\frac{3\pi}{2}$, $\pm\,\frac{5\pi}{2}$, $...$.
The \textbf{range}\index{range} of a function $f(x)$ is the set of all values that
$f(x)$ can take over its domain. For example, the range of $f(x)=\sin\;x$ is the set of all real
numbers between $-1$ and $1$ (i.e. the interval $\ival{-1}{1}$), whereas the range of $f(x) =
\tan\;x$ is the set of all real numbers, as we can see from their graphs.

A function $f(x)$ is \textbf{periodic} if there exists a number $p>0$ such that $x+p$ is
in the domain of $f(x)$ whenever $x$ is, and if the following relation holds:
\begin{equation}\label{eqn:periodic}
 f(x+p) ~=~ f(x) \quad\text{for all $x$}
\end{equation}
There could be many numbers $p$ that satisfy the above requirements. If there is a smallest such
number $p$, then we call that number the \textbf{period}\index{period of a function} of the
function $f(x)$.

\begin{exmp}
 The functions $\sin\;x$, $\cos\;x$, $\csc\;x$, and $\sec\;x$ all have the same period: $2\pi$
 radians. We
 saw in Section 5.1 that the graphs of $y=\tan\;x$ and $y=\cot\;x$ repeat every $2\pi$ radians
 but they also repeat every $\pi$ radians. Thus, the functions $\tan\;x$ and $\cot\;x$ have a period
 of $\pi$ radians.
\end{exmp}
\begin{exmp}
 What is the period of $f(x)=\sin\;2x\,$?\vspace{1mm}
 \par\noindent\textbf{Solution:} The graph of $y=\sin\;2x$ is shown in Figure \ref{fig:sine2x}, along
 with the graph of $y=\sin\;x$ for comparison, over the interval $\ival{0}{2\pi}$.
 Note that $\sin\;2x$ ``goes twice as fast'' as $\sin\;x$.

\begin{figure}[ht]
 \begin{center}
  \begin{tikzpicture}[scale=1.2,every node/.style={font=\small}]
   \begin{scope}[shift={(3,0)},dashed,line width=1pt,x=6cm/360]
    \draw[black!60,line width=0.3pt,dotted] (0,-1) grid[xstep=45,ystep=1] (360,1);
	\draw[black!60,line width=0.3pt,-latex] (0,0) -- (380,0) node[right] {$x$};
	\draw[black!60,line width=0.3pt,-latex] (0,-1.2) -- (0,1.5) node[above] {$y$};
	\pgfplothandlerlineto
	\pgfplotfunction{\x}{0,5,...,360}{\pgfpointxy{\x}{sin(\x)}}
	\pgfusepath{stroke}
	\node[black,left] at (0,0) {$0$};
	\foreach \pos in {45,90,135,180,225,270,315,360}
	 \draw[black!60,line width=0.3pt,shift={(\pos,0)}] (0pt,3pt) -- (0pt,-3pt);
	\foreach \pos in {-1,1}
	 \draw[black!60,line width=0.3pt,shift={(0,\pos)}] (3pt,0pt) -- (-3pt,0pt) node[black,left]
      {$\pos$};
	\node[black,below] at (45,-0.1) {$\tfrac{\pi}{4}$};
	\node[black,below] at (90,-0.1) {$\tfrac{\pi}{2}$};
	\node[black,below] at (135,-0.1) {$\tfrac{3\pi}{4}$};
	\node[black,below] at (180,-0.1) {$\pi$};
	\node[black,below] at (225,-0.1) {$\tfrac{5\pi}{4}$};
	\node[black,below] at (270,-0.1) {$\tfrac{3\pi}{2}$};
	\node[black,below] at (315,-0.1) {$\tfrac{7\pi}{4}$};
	\node[black,below] at (360,-0.1) {$2\pi$};
   \end{scope}
   \begin{scope}[shift={(3,0)},color=linecolor,line width=1.5pt,x=6cm/360]
	\pgfplothandlerlineto
	\pgfplotfunction{\x}{0,5,...,360}{\pgfpointxy{\x}{sin(2*\x)}}
	\pgfusepath{stroke}
   \end{scope}
  \draw [linecolor,line width=1.5pt] (0,1) -- (0.5,1) node[black,right] {$y=\sin\;2x$};
  \draw [dashed,line width=1pt] (0,0.5) -- (0.5,0.5) node[right] {$y=\sin\;x$};
  \end{tikzpicture}\vspace{-6mm}
 \end{center}
 \caption[]{\quad Graph of $y=\sin\;2x$}
 \label{fig:sine2x}
\end{figure}

 For example, for $x$ from $0$ to $\frac{\pi}{2}$, $\sin\;x$ goes from $0$ to $1$, but $\sin\;2x$
 is able to go from $0$ to $1$ quicker, just over the interval $\ival{0}{\frac{\pi}{4}}$.
 While $\sin\;x$ takes a full $2\pi$ radians to go through an entire \emph{cycle}\index{cycle} (the
 largest part of the graph that does not repeat), $\sin\;2x$ goes through an entire cycle in just
 $\pi$ radians. So the period of $\sin\;2x$ is $\pi$ radians.
\end{exmp}
\divider
\newpage
The above example made use of the graph of $\sin\;2x$, but the period can be found analytically.
Since $\sin\;x$ has period $2\pi$,\footnote{We will usually leave out the ``radians'' part when
discussing periods from now on.} we know that $\sin\;(x+2\pi) = \sin\;x$ for all $x$. Since $2x$
is a number for all $x$, this means in particular that $\sin\;(2x+2\pi) = \sin\;2x$ for all $x$.
Now define $f(x)=\sin\;2x$. Then
\begin{align*}
 f(x+\pi) ~&=~ \sin\;2\,(x+\pi)\\
 &=~ \sin\;(2x+2\pi)\\
 &=~ \sin\;2x \quad\text{(as we showed above)}\\
 &=~ f(x)
\end{align*}
for all $x$, so the period $p$ of $\sin\;2x$ is \emph{at most} $\pi$, by our definition of period.
We have to show that $p>0$ can not be smaller than $\pi$. To do this, we will use a \emph{proof by
contradiction}. That is, assume that $0<p<\pi$, then show that this leads to some
contradiction, and hence can not be true. So suppose $0<p<\pi$. Then $0<2p<2\pi$, and hence
\begin{align*}
 \sin\;2x ~&=~ f(x)\\
 &=~ f(x+p) \quad\text{(since $p$ is the period of $f(x)$)}\\
 &=~ \sin\;2(x+p)\\
 &=~ \sin\;(2x+2p)
\end{align*}
for all $x$. Since any number $u$ can be written as $2x$ for some $x$ (i.e $u = 2(u/2)$), this
means that $\sin\;u = \sin\;(u+2p)$ for all real numbers $u$, and hence the period of $\sin\;x$ is
as most $2p$. This is a contradiction. Why? Because the period of $\sin\;x$ is $2\pi > 2p$. Hence,
the period $p$ of $\sin\;2x$ can not be less than $\pi$, so the period must equal $\pi$.

The above may seem like a lot of work to prove something that was visually obvious from the graph
(and intuitively obvious by the ``twice as fast'' idea). Luckily, we do not need to go through all
that work for each function, since a similar argument works when $\sin\;2x$ is replaced by
$\sin\;\omega x$ for any positive real number $\omega$: instead of dividing $2\pi$ by $2$ to get
the period, divide by $\omega$. And the argument works for the other trigonometric functions as
well. Thus, we get:

\begin{center}\statecomment{For any number $\omega >0$:
 \begin{alignat*}{4}
  \sin\;\omega x ~~&\text{has period}~~ \frac{2\pi}{\omega}
   \qquad\quad&\csc\;\omega x ~~&\text{has period}~~ \frac{2\pi}{\omega}\\[2pt]
  \cos\;\omega x ~~&\text{has period}~~ \frac{2\pi}{\omega}
   \qquad\quad&\sec\;\omega x ~~&\text{has period}~~ \frac{2\pi}{\omega}\\[2pt]
  \tan\;\omega x ~~&\text{has period}~~ \frac{\pi}{\omega}
   \qquad\quad&\cot\;\omega x ~~&\text{has period}~~ \frac{\pi}{\omega}
 \end{alignat*}
}\end{center}

If $\omega < 0$, then use $\sin\;(-A) = -\sin\;A$ and $\cos\;(-A) = \cos\;A$ (e.g.
$\sin\;(-3x) = -\sin\;3x$).

\newpage
\begin{exmp}
 The period of $y=\cos\;3x$ is $\frac{2\pi}{3}$ and the period of $y=\cos\;\frac{1}{2}x$ is $4\pi$.
 The graphs of both functions are shown in Figure \ref{fig:cosine3x}:

\begin{figure}[ht]
 \begin{center}
  \begin{tikzpicture}[scale=1.1,every node/.style={font=\small}]
   \begin{scope}[shift={(0,0)},dashed,line width=1pt,x=6cm/360]
    \draw[black!60,line width=0.3pt,dotted] (0,-1) grid[xstep=45,ystep=1] (720,1);
	\draw[black!60,line width=0.3pt,-latex] (0,0) -- (740,0) node[right] {$x$};
	\draw[black!60,line width=0.3pt,-latex] (0,-1.2) -- (0,1.5) node[above] {$y$};
	\pgfplothandlerlineto
	\pgfplotfunction{\x}{0,5,...,720}{\pgfpointxy{\x}{cos(3*\x)}}
	\pgfusepath{stroke}
	\node[black,left] at (0,0) {$0$};
	\foreach \pos in {30,60,90,120,150,180,210,240,270,300,330,360,390,420,450,480,510,540,570,600,
	 630,660,690,720}
	 \draw[black!60,line width=0.3pt,shift={(\pos,0)}] (0pt,3pt) -- (0pt,-3pt);
	\foreach \pos in {-1,1}
	 \draw[black!60,line width=0.3pt,shift={(0,\pos)}] (3pt,0pt) -- (-3pt,0pt) node[black,left]
      {$\pos$};
	\node[black,below] at (30,-0.1) {$\tfrac{\pi}{6}$};
	\node[black,below] at (60,-0.1) {$\tfrac{\pi}{3}$};
	\node[black,below] at (90,-0.1) {$\tfrac{\pi}{2}$};
	\node[black,below] at (120,-0.1) {$\tfrac{2\pi}{3}$};
	\node[black,below] at (150,-0.1) {$\tfrac{5\pi}{6}$};
	\node[black,below] at (180,-0.1) {$\pi$};
	\node[black,below] at (210,-0.1) {$\tfrac{7\pi}{6}$};
	\node[black,below] at (240,-0.1) {$\tfrac{4\pi}{3}$};
	\node[black,below] at (270,-0.1) {$\tfrac{3\pi}{2}$};
	\node[black,below] at (300,-0.1) {$\tfrac{5\pi}{3}$};
	\node[black,below] at (330,-0.1) {$\tfrac{11\pi}{6}$};
	\node[black,below] at (360,-0.1) {$2\pi$};
	\node[black,below] at (390,-0.1) {$\tfrac{13\pi}{6}$};
	\node[black,below] at (420,-0.1) {$\tfrac{7\pi}{3}$};
	\node[black,below] at (450,-0.1) {$\tfrac{5\pi}{2}$};
	\node[black,below] at (480,-0.1) {$\tfrac{8\pi}{3}$};
	\node[black,below] at (510,-0.1) {$\tfrac{17\pi}{6}$};
	\node[black,below] at (540,-0.1) {$3\pi$};
	\node[black,below] at (570,-0.1) {$\tfrac{19\pi}{6}$};
	\node[black,below] at (600,-0.1) {$\tfrac{10\pi}{3}$};
	\node[black,below] at (630,-0.1) {$\tfrac{7\pi}{2}$};
	\node[black,below] at (660,-0.1) {$\tfrac{11\pi}{3}$};
	\node[black,below] at (690,-0.1) {$\tfrac{23\pi}{6}$};
	\node[black,below] at (720,-0.1) {$4\pi$};
   \end{scope}
   \begin{scope}[shift={(0,0)},color=linecolor,line width=1.5pt,x=6cm/360]
	\pgfplothandlerlineto
	\pgfplotfunction{\x}{0,5,...,720}{\pgfpointxy{\x}{cos(0.5*\x)}}
	\pgfusepath{stroke}
   \end{scope}
  \draw [linecolor,line width=1.5pt] (2,2) -- (2.5,2) node[black,right] {$y=\cos\;\frac{1}{2}x$};
  \draw [dashed,line width=1pt] (2,1.5) -- (2.5,1.5) node[right] {$y=\cos\;3x$};
  \end{tikzpicture}\vspace{-6mm}
 \end{center}
 \caption[]{\quad Graph of $y=\cos\;3x$ and $y=\cos\;\frac{1}{2}x$}
 \label{fig:cosine3x}
\end{figure}
\end{exmp}
\divider
\vspace{1mm}

We know that $\;-1 \le \sin\;x \le 1\;$ and $\;-1 \le \cos\;x \le 1\;$ for all $x$. Thus, for a
constant $A \ne 0$,
\begin{displaymath}
 -\abs{A} ~\le~ A\,\sin\;x ~\le~ \abs{A} \quad\text{and}\quad
 -\abs{A} ~\le~ A\,\cos\;x ~\le~ \abs{A}
\end{displaymath}
for all $x$. In this case, we call $\abs{A}$ the \textbf{amplitude}\index{amplitude} of the
functions $y=A\,\sin\;x$ and $y=A\,\cos\;x$. In general, the amplitude of a periodic curve $f(x)$
is half the difference of the largest and smallest values that $f(x)$ can take:
\begin{displaymath}
 \text{Amplitude of $f(x)$} ~=~ \frac{\text{(maximum of $f(x)$)} ~-~ \text{(minimum of $f(x)$)}}{2}
\end{displaymath}
In other words, the amplitude is the distance from either the top or bottom of the curve to the
horizontal line that divides the curve in half, as in Figure \ref{fig:amplitude}.

\begin{figure}[ht]
 \begin{center}
  \begin{tikzpicture}[scale=1.2,every node/.style={font=\small}]
   \begin{scope}[shift={(0,0)},color=linecolor,line width=1.5pt,x=6cm/360]
    \draw[black!60,line width=0.3pt,dotted] (0,-1) grid[xstep=45,ystep=1] (360,1);
	\draw[black!60,line width=0.3pt,-latex] (0,0) -- (380,0) node[right] {$x$};
	\draw[black!60,line width=0.3pt,-latex] (0,-1.2) -- (0,1.5) node[above] {$y$};
	\pgfplothandlerlineto
	\pgfplotfunction{\x}{0,5,...,360}{\pgfpointxy{\x}{sin(2*\x)}}
	\pgfusepath{stroke}
	\node[black,left] at (0,0) {$0$};
	\foreach \pos in {45,90,135,180,225,270,315,360}
	 \draw[black!60,line width=0.3pt,shift={(\pos,0)}] (0pt,3pt) -- (0pt,-3pt);
	\foreach \pos in {-1,1}
	 \draw[black!60,line width=0.3pt,shift={(0,\pos)}] (3pt,0pt) -- (-3pt,0pt);
	\node[black,left] at (0,1) {$\abs{A}$};
	\node[black,left] at (0,-1) {$-\abs{A}$};
	\node[black,below] at (45,-0.1) {$\tfrac{\pi}{4}$};
	\node[black,below] at (90,-0.1) {$\tfrac{\pi}{2}$};
	\node[black,below] at (135,-0.1) {$\tfrac{3\pi}{4}$};
	\node[black,below] at (180,-0.1) {$\pi$};
	\node[black,below] at (225,-0.1) {$\tfrac{5\pi}{4}$};
	\node[black,below] at (270,-0.1) {$\tfrac{3\pi}{2}$};
	\node[black,below] at (315,-0.1) {$\tfrac{7\pi}{4}$};
	\node[black,below] at (360,-0.1) {$2\pi$};
   \end{scope}
   \begin{scope}[>=latex]
    \draw [|<->|] (-1,-1) -- (-1,1) node[midway,fill=white] {$2\,\abs{A}$};
    \draw [|<->|] (7,0) -- (7,1) node[midway,right] {$\abs{A}$};
    \draw [<->|] (7,0) -- (7,-1) node[midway,right] {$\abs{A}$};
   \end{scope}
  \end{tikzpicture}\vspace{-6mm}
 \end{center}
 \caption[]{\quad Amplitude $= \frac{\text{max} - \text{min}}{2} = \frac{\abs{A} - (-\abs{A})}{2} =
  \abs{A}$}
 \label{fig:amplitude}
\end{figure}

Not all periodic curves have an amplitude. For example, $\tan\;x$ has neither a maximum nor a
minimum, so its amplitude is undefined. Likewise, $\cot\;x$, $\csc\;x$, and $\sec\;x$ do not have
an amplitude. Since the amplitude involves vertical distances, it has no effect on the period of
a function, and vice versa.
\newpage
\begin{exmp}
 Find the amplitude and period of $y=3\,\cos\;2x$.\vspace{1mm}
 \par\noindent\textbf{Solution:} The amplitude is $\abs{3} = 3$ and the period is
 $\frac{2\pi}{2}=\pi$. The graph is shown in Figure \ref{fig:exmp3cos2x}:\vspace{-2mm}

\begin{figure}[ht]
 \begin{center}
  \begin{tikzpicture}[scale=1.2,every node/.style={font=\small}]
   \begin{scope}[shift={(0,0)},color=linecolor,line width=1.5pt,x=8cm/360,y=1.5cm/3]
    \draw[black!60,line width=0.3pt,dotted] (0,-3) grid[xstep=45,ystep=1] (360,3);
	\draw[black!60,line width=0.3pt,-latex] (0,0) -- (380,0) node[right] {$x$};
	\draw[black!60,line width=0.3pt,-latex] (0,-3.2) -- (0,3.5) node[above] {$y$};
	\pgfplothandlerlineto
	\pgfplotfunction{\x}{0,5,...,360}{\pgfpointxy{\x}{3*cos(2*\x)}}
	\pgfusepath{stroke}
	\node[black,left] at (0,0) {$0$};
	\foreach \pos in {30,60,90,120,150,180,210,240,270,300,330,360}
	 \draw[black!60,line width=0.3pt,shift={(\pos,0)}] (0pt,3pt) -- (0pt,-3pt);
	\foreach \pos in {-3,-2,-1,1,2,3}
	 \draw[black!60,line width=0.3pt,shift={(0,\pos)}] (3pt,0pt) -- (-3pt,0pt) node[black,left]
      {$\pos$};
	\node[black,below] at (30,-0.1) {$\tfrac{\pi}{6}$};
	\node[black,below] at (60,-0.1) {$\tfrac{\pi}{3}$};
	\node[black,below] at (90,-0.1) {$\tfrac{\pi}{2}$};
	\node[black,below] at (120,-0.1) {$\tfrac{2\pi}{3}$};
	\node[black,below] at (150,-0.1) {$\tfrac{5\pi}{6}$};
	\node[black,below] at (180,-0.1) {$\pi$};
	\node[black,below] at (210,-0.1) {$\tfrac{7\pi}{6}$};
	\node[black,below] at (240,-0.1) {$\tfrac{4\pi}{3}$};
	\node[black,below] at (270,-0.1) {$\tfrac{3\pi}{2}$};
	\node[black,below] at (300,-0.1) {$\tfrac{5\pi}{3}$};
	\node[black,below] at (330,-0.1) {$\tfrac{11\pi}{6}$};
	\node[black,below] at (360,-0.1) {$2\pi$};
   \end{scope}
   \begin{scope}[>=latex]
    \draw [|<->|] (-1,-1.5) -- (-1,1.5) node[midway,fill=white] {$6$};
    \draw [|<->|] (9,0) -- (9,1.5) node[midway,right] {$3$};
    \draw [<->|] (9,0) -- (9,-1.5) node[midway,right] {$3$};
   \end{scope}
  \end{tikzpicture}\vspace{-6mm}
 \end{center}
 \caption[]{\quad $y=3\,\cos\;2x$}
 \label{fig:exmp3cos2x}
\end{figure}
\end{exmp}\vspace{-7mm}
\begin{exmp}
 Find the amplitude and period of $y=2 - 3\,\sin\;\frac{2\pi}{3}x$.\vspace{1mm}
 \par\noindent\textbf{Solution:} The amplitude of $-3\,\sin\;\frac{2\pi}{3}x$ is $\abs{-3} =3$. Adding
 $2$ to that function to get the function
 $y=2 - 3\,\sin\;\frac{2\pi}{3}x$ does not change the amplitude, even
 though it does change the maximum and minimum. It just shifts the entire graph upward by $2$. So
 in this case, we have
 \begin{displaymath}
  \text{Amplitude} ~=~ \frac{\text{max} ~-~ \text{min}}{2} ~=~ \frac{5 ~-~ (-1)}{2} ~=~ \frac{6}{2}
   ~=~ 3 ~.
 \end{displaymath}
 The period is $\dfrac{2\pi}{\frac{2\pi}{3}}=3$. The graph is shown in Figure \ref{fig:exmp2m3sinx}:\vspace{-1mm}

\begin{figure}[ht]
 \begin{center}
  \begin{tikzpicture}[scale=1.2,every node/.style={font=\small}]
   \begin{scope}[shift={(0,0)},color=linecolor,line width=1.5pt,x=8cm/540,y=2cm/3]
    \draw[black!60,line width=0.3pt,dotted] (0,-1) grid[xstep=45,ystep=1] (540,5);
	\draw[black!60,line width=0.3pt,-latex] (0,0) -- (560,0) node[right] {$x$};
	\draw[black!60,line width=0.3pt,-latex] (0,-1.2) -- (0,5.5) node[above] {$y$};
	\pgfplothandlerlineto
	\pgfplotfunction{\x}{0,5,...,540}{\pgfpointxy{\x}{2-3*sin(2*\x/3)}}
	\pgfusepath{stroke}
	\node[black,left] at (0,0) {$0$};
	\foreach \pos in {135,270,405,540}
	 \draw[black!60,line width=0.3pt,shift={(\pos,0)}] (0pt,3pt) -- (0pt,-3pt);
	\foreach \pos in {-1,1,2,3,4,5}
	 \draw[black!60,line width=0.3pt,shift={(0,\pos)}] (3pt,0pt) -- (-3pt,0pt) node[black,left]
      {$\pos$};
	\node[black,below] at (135,-0.1) {$\tfrac{3}{4}$};
	\node[black,below] at (270,-0.1) {$\tfrac{3}{2}$};
	\node[black,below] at (405,-0.1) {$\tfrac{9}{4}$};
	\node[black,below] at (540,-0.1) {$3$};
    \draw[dashed,black!60,line width=0.5pt] (0,2) -- (540,2);
   \end{scope}
   \begin{scope}[>=latex]
    \draw [|<->|] (-1,-0.67) -- (-1,3.33) node[midway,fill=white] {$6$};
    \draw [|<->|] (9,1.33) -- (9,3.33) node[midway,right] {$3$};
    \draw [<->|] (9,1.33) -- (9,-0.67) node[midway,right] {$3$};
   \end{scope}
  \end{tikzpicture}\vspace{-6mm}
 \end{center}
 \caption[]{\quad $y=2-3\,\sin\;\frac{2\pi}{3}x$}
 \label{fig:exmp2m3sinx}
\end{figure}
\end{exmp}\vspace{-5mm}
\divider
\vspace{1mm}

So far in our examples we have been able to determine the amplitudes of sinusoidal curves fairly
easily. This will not always be the case.
\newpage
\begin{exmp}\label{exmp:3sinx4cosx}
  Find the amplitude and period of $y=3\,\sin\;x + 4\,\cos\;x$.\vspace{1mm}
 \par\noindent\textbf{Solution:} This is sometimes called a \emph{combination} sinusoidal curve, since
 it is the sum of two such curves. The period is still simple to determine: since
 $\sin\;x$ and $\cos\;x$ each repeat every $2\pi$ radians, then so does the combination
 $3\,\sin\;x + 4\,\cos\;x$. Thus, $y=3\,\sin\;x + 4\,\cos\;x$ has period $2\pi$. We can see this in
 the graph, shown in Figure \ref{fig:exmp3sinx4cosx}:\vspace{-1mm}

\begin{figure}[ht]
 \begin{center}
   \input{3sinx4cosx.tex}\vspace{-6mm}
 \end{center}
 \caption[]{\quad $y=3\,\sin\;x + 4\,\cos\;x$}
 \label{fig:exmp3sinx4cosx}
\end{figure}

The graph suggests that the amplitude is $5$, which may not be immediately obvious just by looking at
how the function is defined. In fact, the definition $y=3\,\sin\;x + 4\,\cos\;x$ may tempt you to
think that the amplitude is $7$, since the largest that $3\,\sin\;x$ could be is $3$ and the largest
that $4\,\cos\;x$ could be is $4$, so that the largest their sum could be is $3+4=7$. However,
$3\,\sin\;x$ can never equal $3$ for the same $x$ that makes $4\,\cos\;x$ equal to $4$ (why?).

\piccaption[]{\label{fig:tri345}}\parpic[r]{\begin{tikzpicture}[scale=0.5,
 every node/.style={font=\small}]
 \fill [fill=fillcolor] (0,0) -- (3,0) -- (3,4) -- (0,0);
 \draw (0:1.5) arc (0:53.13:1.5);
 \draw [line width=0.5pt] (2.625,0) -- (2.625,0.375) -- (3,0.375);
 \draw [linecolor,line width=1.5pt] (0,0) -- (3,0) -- (3,4) -- cycle;
 \node [below] at (1.5,0) {$3$};
 \node [right] at (3,2) {$4$};
 \node [above left] at (1.5,2) {$5$};
 \node at (0.9,0.4) {$\theta$};
\end{tikzpicture}}
\picskip{4}
There is a useful technique (which we will discuss further in Chapter 6) for showing that the
amplitude of $y=3\,\sin\;x + 4\,\cos\;x$ is $5$. Let $\theta$ be the angle shown in the right
triangle in Figure \ref{fig:tri345}. Then $\cos\;\theta = \frac{3}{5}$ and $\sin\;\theta =
\frac{4}{5}$. We can use this as follows:
\begin{align*}
 y ~&=~ 3\,\sin\;x ~+~ 4\,\cos\;x\\
 &=~ 5\,\left( \tfrac{3}{5}\,\sin\;x ~+~ \tfrac{4}{5}\,\cos\;x \right)\\
 &=~ 5\,( \cos\;\theta\;\sin\;x ~+~ \sin\;\theta\;\cos\;x )\\
 &=~ 5\,\sin\;(x+\theta)\quad\text{(by the sine addition formula)}
\end{align*}
Thus, $\abs{y} = \abs{5\,\sin\;(x+\theta)} = \abs{5}\,\cdot\,\abs{\sin\;(x+\theta)} \le (5)(1) = 5$,
so the amplitude of $y=3\,\sin\;x + 4\,\cos\;x$ is $5$.
\end{exmp}\vspace{-3mm}
\divider\vspace{-2mm}
\newpage
In general, a combination of sines and cosines will have a period equal to the \emph{lowest common
multiple} of the periods of the sines and cosines being added. In Example \ref{exmp:3sinx4cosx},
$\sin\;x$ and $\cos\;x$ each have period $2\pi$, so the lowest common multiple (which is always an
\emph{integer} multiple) is $1 \,\cdot\, 2\pi = 2\pi$.

\begin{exmp}\label{exmp:cos6xsin4x}
 Find the period of $y=\cos\;6x + \sin\;4x$.\vspace{1mm}
 \par\noindent\textbf{Solution:} The period of $\cos\;6x$ is $\frac{2\pi}{6} = \frac{\pi}{3}$, and the
 period of $\sin\;4x$ is $\frac{2\pi}{4} = \frac{\pi}{2}$. The lowest common multiple of
 $\frac{\pi}{3}$ and $\frac{\pi}{2}$ is $\pi$:
 \begin{alignat*}{4}
  1 \;\cdot\; \tfrac{\pi}{3} ~&=~ \tfrac{\pi}{3} \quad\quad\quad
   &1 \;&\cdot\; \tfrac{\pi}{2} ~&=~ \tfrac{\pi}{2}\\
  2 \;\cdot\; \tfrac{\pi}{3} ~&=~ \tfrac{2\pi}{3} \quad\quad\quad
   &2 \;&\cdot\; \tfrac{\pi}{2} ~&=~ \pi\\
  3 \;\cdot\; \tfrac{\pi}{3} ~&=~ \pi \quad\quad\quad &{} &{}\\
 \end{alignat*}
 Thus, the period of $y=\cos\;6x + \sin\;4x$ is $\pi$. We can see this from its graph in Figure
 \ref{fig:exmpcos6xsin4x}:
 
\begin{figure}[ht]
 \begin{center}
   \input{cos6xsin4x.tex}\vspace{-6mm}
 \end{center}
 \caption[]{\quad $y=\cos\;6x + \sin\;4x$}
 \label{fig:exmpcos6xsin4x}
\end{figure}

 What about the amplitude? Unfortunately we can not use the technique from Example
 \ref{exmp:3sinx4cosx}, since we are not taking the cosine and sine of the same angle; we are
 taking the cosine of $6x$ but the sine of $4x$. In this case, it appears from the graph that the
 maximum is close to $2$ and the minimum is close to $-2$. In Chapter 6, we will describe how to
 use a numerical computation program to show that the maximum and minimum are
 $\pm\,1.90596111871578$, respectively (accurate to within $\approx 2.2204 \times 10^{-16}$). Hence,
 the amplitude is $1.90596111871578$.
\end{exmp}\vspace{-3mm}
\divider\vspace{-2mm}
\newpage
Generalizing Example \ref{exmp:3sinx4cosx}, an expression of the form
$a\,\sin\;\omega x \;+\; b\,\cos\;\omega x$ is equivalent to
$\sqrt{a^2 + b^2}\;\sin\;(x+\theta)$, where $\theta$ is an angle such that
$\cos\;\theta = \frac{a}{\sqrt{a^2 + b^2}}$ and $\sin\;\theta = \frac{b}{\sqrt{a^2 + b^2}}$. So
$y=a\,\sin\;\omega x \;+\; b\,\cos\;\omega x$ will have amplitude $\sqrt{a^2 + b^2}$. Note that this
method only works when the angle $\omega x$ is the same in both the sine and cosine terms.

We have seen how adding a constant to a function shifts the entire graph vertically. We will now see
how to shift the entire graph of a periodic curve horizontally.

\piccaption[]{\enskip $y=A\,\sin\;\omega x$\label{fig:phasenone}}\parpic[r]{\begin{tikzpicture}[scale=1.2,
 every node/.style={font=\small}]
 \begin{scope}[shift={(0,0)},color=linecolor,line width=1.5pt,x=3cm/360]
  \draw[black!60,line width=0.3pt,dotted] (0,-1) grid[xstep=90,ystep=1] (360,1);
  \draw[black!60,line width=0.3pt,-latex] (0,0) -- (380,0) node[right] {$x$};
  \draw[black!60,line width=0.3pt,-latex] (0,-1.2) -- (0,1.6) node[above] {$y$};
  \pgfplothandlerlineto
  \pgfplotfunction{\x}{0,5,...,360}{\pgfpointxy{\x}{sin(\x)}}
  \pgfusepath{stroke}
  \node[black,left] at (0,0) {$0$};
  \foreach \pos in {90,180,270,360}
   \draw[black!60,line width=0.3pt,shift={(\pos,0)}] (0pt,3pt) -- (0pt,-3pt);
  \foreach \pos in {-1,1}
   \draw[black!60,line width=0.3pt,shift={(0,\pos)}] (3pt,0pt) -- (-3pt,0pt);
  \node[black,left] at (0,1) {$A$};
  \node[black,left] at (0,-1) {$-A$};
  \node[black,below] at (180,-0.1) {$\tfrac{\pi}{\omega}$};
  \node[black,below] at (360,-0.1) {$\tfrac{2\pi}{\omega}$};
 \end{scope}
 \begin{scope}[>=latex]
  \draw [<->|] (0,1.3) -- (3,1.3) node[midway,fill=white] {period $= \tfrac{2\pi}{\omega}$};
 \end{scope}
\end{tikzpicture}}
Consider a function of the form $y=A\,\sin\;\omega x$, where $A$ and $\omega$ are nonzero constants.
For simplicity we will assume that $A >0$ and $\omega > 0$ (in general either one could be
negative). Then the amplitude is $A$ and the period is $\frac{2\pi}{\omega}$. The graph is shown in
Figure \ref{fig:phasenone}.

Now consider the function $y=A\,\sin\;(\omega x + \phi)$, where $\phi$ is some constant. The
amplitude is still $A$, and the period is still $\frac{2\pi}{\omega}$, since $\omega x + \phi$ is a
linear function of $x$. Also, we know that the sine function goes through an entire cycle when its
angle goes from $0$ to $2\pi$. Here, we are taking the sine of the angle $\omega x + \phi$. So as
$\omega x + \phi$ goes from $0$ to $2\pi$, an entire cycle of the function
$y=A\,\sin\;(\omega x + \phi)$ will be traced out. That cycle starts when
\begin{align*}
 \omega x + \phi ~=~ 0 \quad&\Rightarrow\quad x ~=~ -\frac{\phi}{\omega}\\
\intertext{and ends when}
 \omega x + \phi ~=~ 2\pi \quad&\Rightarrow\quad x ~=~ \frac{2\pi}{\omega}\;-\;\frac{\phi}{\omega}~.
\end{align*}
Thus, the graph of $y=A\,\sin\;(\omega x + \phi)$ is just the graph of $y=A\,\sin\;\omega x$
shifted horizontally by $\frac{-\phi}{\omega}$, as in Figure \ref{fig:phaseshift}. The quantity $\phi$ is the phase. The graph is
shifted to the right when $\phi <0$, and to the left when $\phi >0$. The amount
$-\frac{\phi}{\omega}$ of the shift is called the \textbf{phase shift}\index{phase shift} of the
graph.  The phase has units of radians, whereas the phase shift has the same units as the $x$ axis.

\begin{figure}[ht]
 \centering
 \subfloat[][ $\phi <0$: right shift]{
  \begin{tikzpicture}[scale=1.2,every node/.style={font=\small}]
   \begin{scope}[shift={(0,0)},color=linecolor,line width=1.5pt,x=3cm/360]
	\draw[black!60,line width=0.3pt,-latex] (0,0) -- (500,0) node[right] {$x$};
	\draw[black!60,line width=0.3pt,-latex] (0,-1.2) -- (0,1.7) node[above] {$y$};
	\pgfplothandlerlineto
	\pgfplotfunction{\x}{90,95,...,450}{\pgfpointxy{\x}{sin(-90+\x)}}
	\pgfusepath{stroke}
	\node[black,left] at (0,0) {$0$};
	\foreach \pos in {90,180,270,360,450}
	 \draw[black!60,line width=0.3pt,shift={(\pos,0)}] (0pt,3pt) -- (0pt,-3pt);
	\foreach \pos in {-1,1}
	 \draw[black!60,line width=0.3pt,shift={(0,\pos)}] (3pt,0pt) -- (-3pt,0pt);
	\node[black,left] at (0,1) {$A$};
	\node[black,left] at (0,-1) {$-A$};
	\node[black,below right] at (425,-0.1) {$\tfrac{2\pi}{\omega}+\tfrac{\phi}{\omega}$};
	\node[black,below] at (90,-0.1) {$\tfrac{\phi}{\omega}$};
   \end{scope}
   \begin{scope}[>=latex]
    \draw [|<->|] (0.75,1.4) -- (3.75,1.4) node[midway,fill=white]
	 {period $= \tfrac{2\pi}{\omega}$};
    \draw [<->|] (0,-0.9) -- (0.75,-0.9);
	\node[below right] at (0,-0.9) {phase shift};
   \end{scope}
  \end{tikzpicture}}
 \qquad\qquad
 \subfloat[][ $\phi >0$: left shift]{
  \begin{tikzpicture}[scale=1.2,every node/.style={font=\small}]
   \begin{scope}[shift={(0,0)},color=linecolor,line width=1.5pt,x=3cm/360]
	\draw[black!60,line width=0.3pt,-latex] (-120,0) -- (320,0) node[right] {$x$};
	\draw[black!60,line width=0.3pt,-latex] (0,-1.2) -- (0,1.7) node[above] {$y$};
	\pgfplothandlerlineto
	\pgfplotfunction{\x}{-90,-85,...,270}{\pgfpointxy{\x}{sin(90+\x)}}
	\pgfusepath{stroke}
	\node[black,below right] at (0,0) {$0$};
	\foreach \pos in {-90,90,180,270}
	 \draw[black!60,line width=0.3pt,shift={(\pos,0)}] (0pt,3pt) -- (0pt,-3pt);
	\foreach \pos in {-1,1}
	 \draw[black!60,line width=0.3pt,shift={(0,\pos)}] (3pt,0pt) -- (-3pt,0pt);
	\node[black,left] at (0,1.1) {$A$};
	\node[black,right] at (0,-1) {$-A$};
	\node[black,below right] at (245,-0.1) {$\tfrac{2\pi}{\omega}+\tfrac{\phi}{\omega}$};
	\node[black,below] at (-90,-0.1) {$\tfrac{\phi}{\omega}$};
   \end{scope}
   \begin{scope}[>=latex]
    \draw [|<->|] (-0.75,1.4) -- (2.25,1.4) node[pos=0.6,fill=white]
	 {period $= \tfrac{2\pi}{\omega}$};
    \draw [|<->] (-0.75,-0.9) -- (0,-0.9);
	\node[below left] at (0,-0.9) {phase shift};
   \end{scope}
  \end{tikzpicture}}\vspace{-2mm}
 \caption[]{\quad Phase shift for $y=A\,\sin\;(\omega x + \phi)$}
 \label{fig:phaseshift}
\end{figure}
\newpage
The phase shift is defined similarly for the other trigonometric functions.

\begin{exmp}
 Find the amplitude, period, and phase shift of $y=3\,\cos\;(2x - \pi)$.\vspace{1mm}
 \par\noindent\textbf{Solution:} The amplitude is $3$, the period is $\frac{2\pi}{2} = \pi$, and the
 phase shift is $\frac{\pi}{2}$. The graph is shown in Figure \ref{fig:exmp3cos2mpi}:\vspace{-1mm}

\begin{figure}[ht]
 \begin{center}
  \begin{tikzpicture}[scale=1.2,every node/.style={font=\small}]
   \begin{scope}[shift={(0,0)},color=linecolor,line width=1.5pt,x=8cm/360,y=1.5cm/3]
    \draw[black!60,line width=0.3pt,dotted] (0,-3) grid[xstep=90,ystep=1] (360,3);
	\draw[black!60,line width=0.3pt,-latex] (0,0) -- (380,0) node[right] {$x$};
	\draw[black!60,line width=0.3pt,-latex] (0,-3.8) -- (0,3.8) node[above] {$y$};
	\pgfplothandlerlineto
	\pgfplotfunction{\x}{0,5,...,360}{\pgfpointxy{\x}{3*cos(-180+2*\x)}}
	\pgfusepath{stroke}
	\node[black,left] at (0,0) {$0$};
	\foreach \pos in {90,180,270,360}
	 \draw[black!60,line width=0.3pt,shift={(\pos,0)}] (0pt,3pt) -- (0pt,-3pt);
	\foreach \pos in {-3,-2,-1,1,2,3}
	 \draw[black!60,line width=0.3pt,shift={(0,\pos)}] (3pt,0pt) -- (-3pt,0pt) node[black,left]
	  {$\pos$};
	\node[black,below] at (90,-0.1) {$\tfrac{\pi}{2}$};
	\node[black,below] at (180,-0.1) {$\pi$};
	\node[black,below] at (270,-0.1) {$\tfrac{3\pi}{2}$};
	\node[black,below] at (360,-0.1) {$2\pi$};
   \end{scope}
   \begin{scope}[>=latex]
    \draw [|<->|] (2,1.9) -- (6,1.9) node[midway,fill=white] {period $= \pi$};
    \draw [<->|] (0,-1.7) -- (2,-1.7);
	\node[below right] at (0,-1.7) {phase shift $= \tfrac{\pi}{2}$};
    \draw [|<->|] (-0.9,0) -- (-0.9,1.5) node[midway,left] {amplitude $= 3$};
   \end{scope}
  \end{tikzpicture}\vspace{-6mm}
 \end{center}
 \caption[]{\quad $y=3\,\cos\;(2x - \pi)$}
 \label{fig:exmp3cos2mpi}
\end{figure}

Notice that the graph is the same as the graph of $y=3\,\cos\;2x$ shifted to the right by
$\frac{\pi}{2}$, the amount of the phase shift.
\end{exmp}
\begin{exmp}
 Find the amplitude, period, and phase shift of $y=-2\,\sin\;\left(3x +
 \frac{\pi}{2}\right)$.\vspace{1mm}
 \par\noindent\textbf{Solution:} The amplitude is $2$, the period is $\frac{2\pi}{3}$, and the
 phase shift is $\frac{-\frac{\pi}{2}}{3} = -\frac{\pi}{6}$. Notice the negative sign in the phase
 shift, since $3x+\pi=3x-(-\pi)$ is in the form $\omega x - \phi$. The graph is shown in Figure
 \ref{fig:exmpm2sin3ppi2}:\vspace{-1mm}

\begin{figure}[ht]
 \begin{center}
  \begin{tikzpicture}[scale=1.1,every node/.style={font=\small}]
   \begin{scope}[shift={(0,0)},color=linecolor,line width=1.5pt,x=8cm/240,y=1.5cm/2]
    \draw[black!60,line width=0.3pt,dotted] (-30,-2) grid[xstep=30,ystep=1] (240,2);
	\draw[black!60,line width=0.3pt,-latex] (-50,0) -- (250,0) node[right] {$x$};
	\draw[black!60,line width=0.3pt,-latex] (0,-2.8) -- (0,2.8) node[above] {$y$};
	\pgfplothandlerlineto
	\pgfplotfunction{\x}{-30,-25,...,240}{\pgfpointxy{\x}{-2*sin(90+3*\x)}}
	\pgfusepath{stroke}
	\node[black,below left] at (0,0) {$0$};
	\foreach \pos in {-30,30,60,90,120,150,180,210,240}
	 \draw[black!60,line width=0.3pt,shift={(\pos,0)}] (0pt,3pt) -- (0pt,-3pt);
	\foreach \pos in {-2,-1,1,2}
	 \draw[black!60,line width=0.3pt,shift={(0,\pos)}] (3pt,0pt) -- (-3pt,0pt) node[black,left]
	  {$\pos$};
	\node[black,below] at (-35,-0.1) {$-\tfrac{\pi}{6}$};
	\node[black,below] at (30,-0.1) {$\tfrac{\pi}{6}$};
	\node[black,below] at (60,-0.1) {$\tfrac{\pi}{3}$};
	\node[black,below] at (90,-0.1) {$\tfrac{\pi}{2}$};
	\node[black,below] at (120,-0.1) {$\tfrac{2\pi}{3}$};
	\node[black,below] at (150,-0.1) {$\tfrac{5\pi}{6}$};
	\node[black,below] at (180,-0.1) {$\pi$};
	\node[black,below] at (210,-0.1) {$\tfrac{7\pi}{6}$};
	\node[black,below] at (240,-0.1) {$\tfrac{4\pi}{3}$};
   \end{scope}
   \begin{scope}[>=latex]
    \draw [|<->|] (-1,1.9) -- (3,1.9) node[pos=0.55,fill=white] {period $= \frac{2\pi}{3}$};
    \draw [|<->] (-1,-1.7) -- (0,-1.7);
	\node[below left] at (0,-1.75) {phase shift $= -\tfrac{\pi}{6}$};
    \draw [|<->|] (-1.1,0) -- (-1.1,1.5) node[midway,left] {amplitude $= 2$};
   \end{scope}
  \end{tikzpicture}\vspace{-6mm}
 \end{center}
 \caption[]{\quad $y=-2\,\sin\;\left( 3x + \frac{\pi}{2} \right)$}
 \label{fig:exmpm2sin3ppi2}
\end{figure}
\end{exmp}\vspace{-4mm}
\divider\vspace{-2mm}
\newpage
In engineering two periodic functions with the same period are said to be \emph{out of
phase} if their phase shifts differ. For example, $\sin\;\left( x -
\frac{\pi}{6} \right)$ and $\sin\;x$ would be $\frac{\pi}{6}$ radians (or $30\Degrees$) out of
phase, and $\sin\;x$ would be said to \emph{lag} $\sin\;\left( x - \frac{\pi}{6} \right)$ by
$\frac{\pi}{6}$ radians, while $\sin\;\left( x - \frac{\pi}{6} \right)$ \emph{leads}
$\sin\;x$ by $\frac{\pi}{6}$ radians. Periodic functions with the same period and the same phase
shift are \emph{in phase}.\index{phase, out of or in}

The following is a summary of the properties of trigonometric graphs:

\begin{center}\statecomment{For any constants $A \ne 0$, $\omega \ne 0$, and $\phi$:
\begin{align*}
 y = A\,\sin\;(\omega x + \phi) ~~&\text{has amplitude $\abs{A}$, period $\tfrac{2\pi}{\omega}$, and
  phase shift $\tfrac{-\phi}{\omega}$}\\
 y = A\,\cos\;(\omega x + \phi) ~~&\text{has amplitude $\abs{A}$, period $\tfrac{2\pi}{\omega}$, and
  phase shift $\tfrac{-\phi}{\omega}$}\\
 y = A\,\tan\;(\omega x + \phi) ~~&\text{has undefined amplitude, period $\tfrac{\pi}{\omega}$, and
  phase shift $\tfrac{-\phi}{\omega}$}\\
 y = A\,\csc\;(\omega x + \phi) ~~&\text{has undefined amplitude, period $\tfrac{2\pi}{\omega}$, and
  phase shift $\tfrac{-\phi}{\omega}$}\\
 y = A\,\sec\;(\omega x + \phi) ~~&\text{has undefined amplitude, period $\tfrac{2\pi}{\omega}$, and
  phase shift $\tfrac{-\phi}{\omega}$}\\
 y = A\,\cot\;(\omega x + \phi) ~~&\text{has undefined amplitude, period $\tfrac{\pi}{\omega}$, and
  phase shift $\tfrac{-\phi}{\omega}$}
\end{align*}}\end{center}\vspace{-4mm}

\divider
\vspace{2mm}

\startexercises\label{sec5dot2}
\vspace{5mm}
{\small
\par\noindent For Exercises 1-12, find the amplitude, period, and phase shift of the given function.
Then graph one cycle of the function, either by hand or by using Gnuplot (see Appendix B).
\begin{enumerate}[\bfseries 1.]
\begin{multicols}{4}
 \item $y=3\,\cos\;\pi x$
 \item $y=\sin\;(2\pi x - \pi)$
 \item $y=-\sin\;(5x + 3)$
 \item $y=1+8\,\cos\;(6x- \pi)$
\end{multicols}
\begin{multicols}{4}
 \item $y=2+\cos\;(5x + \pi)$
 \item $y=1-\sin\;(3\pi - 2x)$
 \item $y=1-\cos\;(3\pi - 2x)$
 \item $y=2\,\tan\;(x - 1)$
\end{multicols}
\begin{multicols}{4}
 \item $y=1-\tan\;(3\pi - 2x)$
 \item $y=\sec\;(2x + 1)$
 \item $y=2\csc\;(2x - 1)$
 \item $y=2+4\,\cot\;(1-x)$
\end{multicols}
 \item For the function $y=2\,\sin\;( x^2 )$, for which values of $x$
  does the function reach its maximum value $2$, and for which values of $x$ does it reach
  its minimum value $-2\,$?
 \item For the function $y=3\,\sin\;x + 4\,\cos\;x$ in Example \ref{exmp:3sinx4cosx}, for which
  values of $x$ does the function reach its maximum value $5$, and for which values of $x$ does it
  reach its minimum value $-5\,$? You can restrict your answers to be between $0$ and $2\pi$.
 \item Graph the function $y=\sin^2 \,x$ from $x=0$ to $x=2\pi$, either by hand or by using Gnuplot.
  What are the amplitude and period of this function?
 \item\label{exer:circuitphase}
  The current $i(t)$ in an AC electrical circuit at time $t\ge 0$ is given by
  $i(t) = I_m \,\sin\;\omega t$, and the voltage $v(t)$ is given by $v(t) = V_m \,\sin\;\omega t$,
  where $V_m > I_m > 0$ and $\omega > 0$ are constants.
  Sketch one cycle of both $i(t)$ and $v(t)$ \emph{together on the same graph} (i.e. on the same set
  of axes). Are the current and voltage in phase or out of phase?
 \item Repeat Exercise \ref{exer:circuitphase} with $i(t)$ the same as before but with $v(t)=
  V_m \,\sin\;\left(\omega t + \frac{\pi}{4}\right)$.
 \item Repeat Exercise \ref{exer:circuitphase} with
  $i(t)=-I_m \,\cos\;\left(\omega t - \frac{\pi}{3}\right)$ and $v(t)=
  V_m \,\sin\;\left(\omega t - \frac{5\pi}{6}\right)$.
\suspend{enumerate}
 For Exercises \ref{exer:ampcombostart}-\ref{exer:ampcomboend}, find the amplitude and period of
 the given function. Then graph one cycle of the function, either by hand or by using Gnuplot.
\resume{enumerate}[{[\bfseries 1.]}]
\begin{multicols}{3}
 \item\label{exer:ampcombostart} $y=3\,\sin\;\pi x \;-\; 5\,\cos\;\pi x$
 \item $y=-5\,\sin\;3x \;+\; 12\,\cos\;3x$
 \item\label{exer:ampcomboend} $y=2\,\cos\;x \;+\; 2\,\sin\;x$
\end{multicols}
 \item Find the amplitude of the function $y=2\,\sin\;( x^2 ) \;+\; \cos\;( x^2 )$.
\suspend{enumerate}
 For Exercises \ref{exer:percombostart}-\ref{exer:percomboend}, find the period of the given
 function. Graph one cycle using Gnuplot.
\resume{enumerate}[{[\bfseries 1.]}]
\begin{multicols}{3}
 \item\label{exer:percombostart} $y=\sin\;3x \;-\; \cos\;5x$
 \item $y=\sin\;\frac{x}{3} \;+\; 2\,\cos\;\frac{3x}{4}$
 \item\label{exer:percomboend} $y=2\,\sin\;\pi x \;+\; 3\,\cos\;\frac{\pi}{3}x$
\end{multicols}
 \item Let $y = 0.5\,\sin\;x ~\sin\;12x\,$. Its graph for $x$ from $0$ to $4\pi$ is shown in
  Figure \ref{fig:modulated}:

\begin{figure}[ht]
 \begin{center}
   \input{modulated.tex}\vspace{-6mm}
 \end{center}
 \caption[]{\quad Modulated wave $y=0.5\,\sin\;x ~\sin\;12x$}
 \label{fig:modulated}
\end{figure}

  You can think of this function as $\sin\;12x$ with a sinusoidally varying ``amplitude''of
  $0.5\,\sin\;x$. What is the period of this function?
  From the graph it looks like the amplitude may be $0.5$.
  Without finding the exact amplitude, explain why the amplitude is in fact \emph{less} than $0.5$.
  The function above is known as a \emph{modulated wave}\index{modulated wave}, and the functions
  $\pm\,0.5\,\sin\;x$ form an \emph{amplitude envelope}\index{amplitude envelope} for the wave (i.e.
  they enclose the wave). Use an identity from Section 3.4 to write this function as a sum of
  sinusoidal curves.
 \item Use Gnuplot to graph the function $y= x^2 \,\sin\;10x$ from $x = -2\pi$ to $x=2\pi$. What
  functions form its amplitude envelope? (Note: Use \texttt{set samples 500} in Gnuplot.)
 \item Use Gnuplot to graph the function $y= \frac{1}{x^2} \,\sin\;80x$ from $x = 0.2$ to $x=\pi$.
  What functions form its amplitude envelope? (Note: Use \texttt{set samples 500} in Gnuplot.)
 \item Does the function $y=\sin\;\pi x \;+\; \cos\;x$ have a period? Explain your answer.
 \item Use Gnuplot to graph the function $y=\frac{\sin\;x}{x}$ from $x=-4\pi$ to $x=4\pi$. What
  happens at $x=0$?
\end{enumerate}}



% % % % % % % % % % % % % % % % % % % % % % % % % % % % % % % % % % % %new section


